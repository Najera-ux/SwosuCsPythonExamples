\documentclass[12pt]{article}
\usepackage{amsmath, amssymb, amsthm}
\usepackage{multicol}
\usepackage{enumitem}
\usepackage{tikz}
\usepackage{xcolor}
\usepackage{lipsum}

\title{Chapter 5.1: Mathematical Induction – The Infinite Ladder of Logic}
\author{SWOSU Discrete Structures}
\date{}

\begin{document}
\maketitle

\section*{Introduction: Climbing the Infinite Ladder}

Suppose we have an \textbf{infinite ladder}. You’re standing at the bottom, holding your coffee in one hand and your sense of dread in the other.  
You know two things:
\begin{enumerate}
  \item You can reach the first rung. (Victory!)
  \item If you can reach any rung $k$, then you can also reach the next rung $k+1$.
\end{enumerate}

\noindent By pure logic (and maybe caffeine), you can conclude that you’ll eventually reach every rung of that ladder.  
\textit{This, my friend, is the essence of mathematical induction.}

\section*{Student Challenge}

Prove that for all positive integers $n$:
\[
1 + 2 + 3 + \dots + n = \frac{n(n+1)}{2}
\]
\vspace{2em}

\noindent \textbf{Hint:} Think of this as a staircase to mathematical greatness.  
Start small (the base case), then prove you can keep going (the inductive step).

\begin{enumerate}[label=\alph*)]
  \item \textbf{Base Case:} Prove it works for $n=1$. (You can do this in your sleep.)
  \item \textbf{Inductive Step:} Assume it works for some $k$, and show it must work for $k+1$.
  \item \textbf{Victory Lap:} Conclude that it works for all $n$. Then treat yourself to a cookie.
\end{enumerate}

\section*{Example: Inductive Thinking in the Wild}

Let’s prove that $n^3 - n$ is divisible by 3 for all positive integers $n$.

\noindent \textbf{Try it yourself:} test $n=1, 2, 3$ — see the pattern?  
Now imagine the induction domino effect at work — if it’s true for one, it’s true for the next, and so on.

\section*{Reflect}

Where in life have you ever used an “inductive” idea —  
believing something is true because it’s true in small cases and keeps working?  
(\textit{Hint:} “My coffee cup always empties itself if I keep drinking” counts.)

\end{document}

