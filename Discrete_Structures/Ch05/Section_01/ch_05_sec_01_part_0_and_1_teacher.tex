\documentclass[12pt]{article}
\usepackage{amsmath, amssymb, amsthm}
\usepackage{multicol}
\usepackage{enumitem}
\usepackage{xcolor}

\title{Teacher Solutions: Induction Worksheet}
\author{SWOSU Discrete Structures}
\date{}

\begin{document}
\maketitle

\section*{Solution: The Ladder Proof}

\noindent \textbf{Base Case:} For $n=1$, LHS = $1$, RHS = $\frac{1(1+1)}{2} = 1$. Works.

\noindent \textbf{Inductive Step:}  
Assume for some $k$,  
\[
1 + 2 + 3 + \dots + k = \frac{k(k+1)}{2}.
\]

\noindent Then for $k+1$:
\[
1 + 2 + 3 + \dots + k + (k+1) = \frac{k(k+1)}{2} + (k+1) = \frac{(k+1)(k+2)}{2}.
\]
Thus, by induction, the statement holds for all positive integers $n$.

\section*{Solution: Divisibility by 3}

For $n^3 - n$:
\[
n^3 - n = n(n-1)(n+1),
\]
which is the product of three consecutive integers — one of them must be divisible by 3.  
Therefore, $n^3 - n$ is divisible by 3 for all $n \in \mathbb{Z}^+$.

\section*{Instructor Note:}
Make this lesson interactive — have students build a “ladder of proofs” on the board.  
Each rung = one $n$. Then ask who can “climb” to the next.

\end{document}

