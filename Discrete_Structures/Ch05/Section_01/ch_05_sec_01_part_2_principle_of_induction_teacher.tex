\documentclass[12pt]{article}
\usepackage{amsmath, amssymb, amsthm}
\usepackage{xcolor}

\title{Teacher Solutions – Section 5.1.2: The Principle of Mathematical Induction}
\author{SWOSU Discrete Structures}
\date{}

\begin{document}
\maketitle

\section*{Solution: Sum of Odd Integers Equals $n^2$}

\textbf{Base Case:} For $n=1$, LHS = $1$, RHS = $1^2 = 1$. True.

\textbf{Inductive Hypothesis:}  
Assume for $n=k$,  
\[
1 + 3 + 5 + \dots + (2k - 1) = k^2.
\]

\textbf{Inductive Step:}  
Then for $n = k+1$:
\[
1 + 3 + 5 + \dots + (2k - 1) + (2(k+1) - 1)
= k^2 + (2k + 1)
= (k+1)^2.
\]
Therefore, by induction, the statement holds for all positive integers $n$.

\section*{Instructor Notes:}
- Reinforce that $P(k)$ is \textit{assumed true only for one integer $k$}, not all integers.
- Emphasize that this is not circular reasoning.
- The domino and ladder metaphors are helpful mental models—keep them visual.
- Let students explain the process in their own metaphors (stairs, cookies, chain reactions, etc.)

\end{document}

