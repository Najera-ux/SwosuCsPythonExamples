\section*{5.1.5 Guidelines for Proofs by Mathematical Induction}

Before diving into examples, it’s helpful to have a template — a “recipe” — for how
to structure a mathematical induction proof. Think of this as your *proof
playbook*, ready for any theorem that starts with “for all $n$...”.

\subsection*{Template for Proofs by Mathematical Induction}

\begin{enumerate}
  \item Express the statement to prove in the form “for all $n \geq b$, $P(n)$” for a fixed integer $b$.
  \item Write \textbf{Basis Step:} Show $P(b)$ is true.
  \item Write \textbf{Inductive Step:} Assume $P(k)$ is true for an arbitrary integer $k \ge b$.
  \item State what needs to be proved: show $P(k+1)$ is true.
  \item Prove $P(k+1)$ using the assumption $P(k)$ — this is your induction magic moment.
  \item Conclude the inductive step with, “This completes the inductive step.”
  \item Finally, close the loop: “By mathematical induction, $P(n)$ is true for all $n \ge b$.”
\end{enumerate}

\begin{center}
\fbox{\parbox{0.85\textwidth}{
\textbf{Quick Tip:} Think of it like knocking over dominos — base case knocks the first one, the inductive step guarantees that each falling domino knocks the next.}}
\end{center}

\subsection*{Student Reflection}
When have you ever used a “domino effect” idea in your own life? Maybe in
a habit, a skill, or a game? Write a few sentences connecting induction to
something you’ve learned by repetition.

