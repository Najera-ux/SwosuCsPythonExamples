\documentclass[12pt]{article}
\usepackage{amsmath,amssymb,amsthm,xcolor}

\title{Teacher Solutions – Section 5.1.3: Strong Induction}
\author{SWOSU Discrete Structures}
\date{}

\begin{document}
\maketitle

\section*{Solution: Every Integer > 1 is a Product of Primes}

\textbf{Base Case:} \(n=2\) is prime — true.

\textbf{Inductive Hypothesis:}  
Assume that for all integers \(2\le n\le k\), each can be written as a product of primes.

\textbf{Inductive Step:}  
For \(k+1\):
\begin{itemize}
    \item If \(k+1\) is prime, we’re done.
    \item If \(k+1=ab\) where \(2\le a,b\le k\), then both \(a\) and \(b\) are products of primes by hypothesis.  
      Therefore, \(k+1=ab\) is a product of primes.
\end{itemize}

Hence, by strong induction, every integer \(n>1\) can be expressed as a product of primes. ∎

\section*{Instructor Notes}
- Emphasize the difference between *regular* and *strong* induction.  
  Students should see that strong induction assumes all previous cases, not just one.
- Great demo: Use Jenga blocks as “integers” — show that pulling one requires all below it to be solid.
- Encourage playful examples (pizza slices, staircases, Pokémon evolutions).

\end{document}

