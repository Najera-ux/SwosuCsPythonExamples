\documentclass[12pt]{article}
\usepackage[margin=1in]{geometry}
\usepackage{amsmath,amssymb}
\setlength{\parskip}{0.6em}
\setlength{\parindent}{0pt}

\newcommand{\Z}{\mathbb{Z}}

\begin{document}
{\large \textbf{Discrete Structures \quad Chapter 4.6 — Cryptography}}

\hrule
\vspace{0.6em}

\section*{Example 11 — RSA as a Partially Homomorphic System}

\subsection*{Concept Refresher: What is “Homomorphic Encryption”?}
A cryptosystem is said to be \emph{homomorphic} if we can perform certain mathematical
operations on encrypted data without decrypting it first.
In simpler words:  
\emph{The math “works through” the encryption.}

For example, suppose encrypting a number $M$ gives $E(M)$.
If the system is \textbf{additively homomorphic}, then  
\[
E(M_1 + M_2) = E(M_1) \cdot E(M_2).
\]
If it is \textbf{multiplicatively homomorphic}, then  
\[
E(M_1 \times M_2) = E(M_1) \times E(M_2).
\]

RSA happens to be \emph{multiplicatively homomorphic}, but not additively so.
Let’s explore why.

\bigskip
\hrule
\vspace{0.5em}

\subsection*{Step-by-Step Walkthrough}

\textbf{Given:}  
Public key $(n,e)$, where $n = pq$ and $e$ is relatively prime to $(p-1)(q-1)$.  
RSA encryption function:
\[
E_{(n,e)}(M) = M^e \bmod n.
\]

Let $M_1$ and $M_2$ be plaintexts such that $0 \le M_1, M_2 < n$.

Then,
\[
E(M_1) \cdot E(M_2)
  = (M_1^e \bmod n)\cdot(M_2^e \bmod n) \bmod n
  = (M_1 M_2)^e \bmod n
  = E(M_1 M_2).
\]

\textbf{Interpretation:}  
Multiplying two ciphertexts corresponds to multiplying their plaintexts before encryption!  
This is the “magic trick” of RSA’s partial homomorphism.

\bigskip
\textbf{Important distinction:}  
\[
E(M_1) + E(M_2) \neq E(M_1 + M_2)
\]
so RSA is not additively homomorphic.  
Only multiplication “passes through” the encryption function.

\bigskip
\hrule
\vspace{0.5em}

\subsection*{Example: Demonstrating RSA’s Multiplicative Homomorphism}

Let $(n,e,d) = (77,7,43)$.  
Encrypt two plaintext messages $M_1 = 5$ and $M_2 = 9$.

\textbf{Step 1.} Encrypt each:
\[
E(5) = 5^7 \bmod 77 = 78125 \bmod 77 = 36.
\]
\[
E(9) = 9^7 \bmod 77 = 4782969 \bmod 77 = 71.
\]

\textbf{Step 2.} Multiply ciphertexts:
\[
E(5)\cdot E(9) \bmod 77 = 36\cdot71 \bmod 77 = 2556 \bmod 77 = 15.
\]

\textbf{Step 3.} Multiply plaintexts and encrypt:
\[
E(5\cdot9) = E(45) = 45^7 \bmod 77 = 15.
\]

They match! 🎉  
So indeed, RSA is multiplicatively homomorphic.

\bigskip
\hrule
\vspace{0.5em}

\section*{Practice Problems}

\textbf{Problem A (Easier).}  
Using $(n,e) = (77,7)$, compute $E(2)$ and $E(3)$, then verify that  
\[
E(2\cdot3) = E(2)\cdot E(3) \pmod{77}.
\]
\emph{Hint:} $E(M) = M^7\bmod77$.  
\vspace{2cm}

\textbf{Problem B (Similar).}  
With $(n,e)=(2537,13)$ (the RSA system from earlier examples),  
let $M_1 = 14$ and $M_2 = 15$.  
Show numerically that  
\[
E(M_1)\cdot E(M_2) \equiv E(M_1M_2) \pmod{2537}.
\]
Use a calculator or write a small Python snippet if needed.  
\vspace{3cm}

\textbf{Problem C (Harder Challenge).}  
Explain, in your own words:
\begin{enumerate}
  \item Why RSA cannot be additively homomorphic.
  \item How this limitation affects using RSA for cloud computations or secure voting.
  \item Why Craig Gentry’s 2009 breakthrough (fully homomorphic encryption) was such a big deal.
\end{enumerate}
\vspace{4cm}

\bigskip
\hrule
\vspace{0.5em}

\section*{Helpful Tips and Intuition}

\begin{itemize}
  \item \textbf{Think of encryption as a “mathematical disguise.”} RSA preserves multiplication but not addition, so we can “multiply in disguise” but not “add in disguise.”
  \item \textbf{Be cautious with modular arithmetic.} When results are large, always reduce modulo $n$ before the next step.
  \item \textbf{Modern context.} Homomorphic encryption allows computation on encrypted data — like asking Google to calculate your taxes without revealing your salary. RSA can’t do that completely, but it was the seed of that dream.
  \item \textbf{Curiosity spark.} Look up Craig Gentry’s 2009 Ph.D. thesis from Stanford — it’s the start of lattice-based, fully homomorphic encryption (FHE).
\end{itemize}

\end{document}

