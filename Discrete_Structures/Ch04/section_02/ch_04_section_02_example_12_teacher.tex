\documentclass[12pt]{article}
\usepackage[margin=1in]{geometry}
\usepackage{amsmath, amssymb}

\begin{document}

\section*{Example 12 Teacher’s Solutions: Fast Modular Exponentiation}

\subsection*{Worked Example}
We already computed:
\[
3^{544} \pmod{645} = 36
\]
with full details shown in the worksheet.

---

\subsection*{Practice Problem Solutions}

\begin{enumerate}
    \item Easier: $2^{13} \pmod{19}$

    Binary expansion of $13 = (1101)_2$.
    Steps:
    \[
    2^1 \equiv 2,\; 2^2 \equiv 4,\; 2^4 \equiv 16,\; 2^8 \equiv 9 \pmod{19}.
    \]
    Multiply relevant powers: $2^8 \cdot 2^4 \cdot 2^1 \equiv 9 \cdot 16 \cdot 2 \equiv 288 \equiv 3 \pmod{19}$.

    Answer: $\boxed{3}$.

    ---

    \item Medium: $7^{45} \pmod{50}$

    Binary expansion of $45 = (101101)_2$.
    Steps:
    \[
    7^1 \equiv 7,\; 7^2 \equiv -1 \equiv 49 \pmod{50}.
    \]
    Notice $7^2 \equiv -1$. Then $7^{44} = (7^2)^{22} \equiv (-1)^{22} \equiv 1 \pmod{50}$.
    Multiply one more factor of $7$: $7^{45} \equiv 7 \pmod{50}$.

    Answer: $\boxed{7}$.

    ---

    \item Harder: $11^{117} \pmod{221}$

    Note: $221 = 13 \cdot 17$. Apply the Chinese Remainder Theorem.

    Mod $13$: $\varphi(13) = 12$. Reduce $117 \equiv 9 \pmod{12}$. So $11^{117} \equiv 11^9 \pmod{13}$.
    Compute: $11 \equiv -2 \pmod{13}$, so $(-2)^9 \equiv -512 \equiv 11 \pmod{13}$.

    Mod $17$: $\varphi(17) = 16$. Reduce $117 \equiv 5 \pmod{16}$. So $11^{117} \equiv 11^5 \pmod{17}$.
    Compute: $11^2 = 121 \equiv 2 \pmod{17}$, $11^4 \equiv 2^2 = 4$, so $11^5 \equiv 11 \cdot 4 = 44 \equiv 10 \pmod{17}$.

    Solve CRT system:
    \[
    x \equiv 11 \pmod{13}, \quad x \equiv 10 \pmod{17}.
    \]
    Answer: $\boxed{142}$.

\end{enumerate}

\end{document}

