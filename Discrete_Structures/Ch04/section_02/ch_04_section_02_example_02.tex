\documentclass[12pt]{article}
\usepackage[margin=1in]{geometry}
\usepackage{amsmath,amssymb}

\begin{document}

\begin{center}
\Large\textbf{Worksheet: Decimal Expansion from Octal (Example 2)}
\end{center}

\section*{Goal}
Convert an octal (base 8) numeral into its decimal (base 10) value using place value.

\section*{Part A — Worked Example (detailed)}
\textbf{Problem.} What is the decimal expansion of $(7016)_8$?

\textbf{Idea.} In base $8$, each position is a power of $8$: from right to left
$8^0,8^1,8^2,8^3,\dots$.
For digits $d_3d_2d_1d_0$ we have:
\[
(d_3d_2d_1d_0)_8 = d_3\cdot 8^3 + d_2\cdot 8^2 + d_1\cdot 8^1 + d_0\cdot 8^0.
\]

\textbf{Step 1: Label digits and place values.}
\[
(7016)_8 = 7\cdot 8^3 + 0\cdot 8^2 + 1\cdot 8^1 + 6\cdot 8^0.
\]

\textbf{Step 2: Evaluate powers of 8.}
\[
8^3=512,\quad 8^2=64,\quad 8^1=8,\quad 8^0=1.
\]

\textbf{Step 3: Multiply digits by powers.}
\[
7\cdot 512 = 3584,\quad 0\cdot 64 = 0,\quad 1\cdot 8 = 8,\quad 6\cdot 1 = 6.
\]

\textbf{Step 4: Add the contributions.}
\[
3584 + 0 + 8 + 6 = 3598.
\]

\textbf{Conclusion.} \fbox{$(7016)_8 = (3598)_{10}$}

\medskip
\textbf{(Optional) Horner’s Method (left-to-right accumulate).}
\[
(((7)\cdot 8 + 0)\cdot 8 + 1)\cdot 8 + 6
= (56\cdot 8 + 1)\cdot 8 + 6 = (449)\cdot 8 + 6 = 3598.
\]
Same answer, fewer big numbers along the way.

\section*{Part B — Easier Practice}
Convert $(52)_8$ to decimal. Show all steps (place values, multiply, add).

\medskip
\textbf{Work:}
\[
(52)_8 = \underline{\hspace{2.5cm}}\cdot 8^1 + \underline{\hspace{2.5cm}}\cdot 8^0
\]
\[
= \underline{\hspace{4cm}} + \underline{\hspace{4cm}} = \underline{\hspace{4cm}}
\]

\vspace{1.5cm}

\section*{Part C — Harder Practice}
Convert $(574321)_8$ to decimal. \emph{Hint:} write powers
$8^5,8^4,8^3,8^2,8^1,8^0$ first.

\medskip
\textbf{Work setup:}
\[
(574321)_8 = 5\cdot 8^5 + 7\cdot 8^4 + 4\cdot 8^3 + 3\cdot 8^2 + 2\cdot 8^1 + 1\cdot 8^0
\]
\[
8^5=\underline{\hspace{1.8cm}},\;
8^4=\underline{\hspace{1.8cm}},\;
8^3=\underline{\hspace{1.8cm}},\;
8^2=\underline{\hspace{1.8cm}},\;
8^1=\underline{\hspace{1.8cm}},\;
8^0=\underline{\hspace{1.8cm}}
\]

\vspace{3cm}

\section*{Quick Self-Check}
Why does the method above look exactly like the base-10 method, except with $8$ instead of $10$?

\end{document}

