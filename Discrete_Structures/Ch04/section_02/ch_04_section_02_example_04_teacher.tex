\documentclass[12pt]{article}
\usepackage[margin=1in]{geometry}
\usepackage{amsmath,amssymb}

\begin{document}

\begin{center}
\Large\textbf{Solutions: Octal Expansion (Example 4)}
\end{center}

\section*{Part A — Worked Example}
Already shown in worksheet:
\[
12345 \div 8 \Rightarrow 1543 \, r1
\]
\[
1543 \div 8 \Rightarrow 192 \, r7
\]
\[
192 \div 8 \Rightarrow 24 \, r0
\]
\[
24 \div 8 \Rightarrow 3 \, r0
\]
\[
3 \div 8 \Rightarrow 0 \, r3
\]
Digits: $(30071)_8$.

---

\section*{Part B — Easier Practice Solutions}

1. $(25)_{10}$:
\[
25 = 8\cdot3 + 1 \quad \Rightarrow r1
\]
\[
3 = 8\cdot0 + 3 \quad \Rightarrow r3
\]
Answer: $(25)_{10} = (31)_{8}$.

\medskip
2. $(64)_{10}$:
\[
64 = 8\cdot8 + 0
\]
\[
8 = 8\cdot1 + 0
\]
\[
1 = 8\cdot0 + 1
\]
Digits: $(100)_{8}$.

\medskip
3. $(255)_{10}$:
\[
255 = 8\cdot31 + 7
\]
\[
31 = 8\cdot3 + 7
\]
\[
3 = 8\cdot0 + 3
\]
Digits: $(377)_{8}$.

---

\section*{Part C — Harder Challenge Solution}

Convert $(54321)_{10}$:

\[
54321 \div 8 = 6790 \, r1
\]
\[
6790 \div 8 = 848 \, r6
\]
\[
848 \div 8 = 106 \, r0
\]
\[
106 \div 8 = 13 \, r2
\]
\[
13 \div 8 = 1 \, r5
\]
\[
1 \div 8 = 0 \, r1
\]

Digits (bottom-to-top): $1,5,2,0,6,1$.

\[
(54321)_{10} = (152061)_{8}.
\]

---

\section*{Teaching Notes}
\begin{itemize}
  \item Emphasize writing quotients and remainders in columns.
  \item Students often forget to read remainders bottom-to-top.
  \item Always check by converting octal back to decimal.
\end{itemize}

\end{document}

