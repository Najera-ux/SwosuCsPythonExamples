\documentclass[12pt]{article}
\usepackage[margin=1in]{geometry}
\usepackage{amsmath,amssymb}

\title{Section 4.2 — Example 9 Teacher Manual\\
Counting Bit Additions in Algorithm 2}
\author{}
\date{}

\begin{document}
\maketitle

\section*{Detailed explanation (matching Example 9)}
At each bit position $j$ Algorithm 2 computes
\[
d=\big\lfloor \tfrac{a_j+b_j+c}{2}\big\rfloor,\qquad s_j=a_j+b_j+c-2d,\qquad c\leftarrow d.
\]
Implementing the three-bit sum $a_j+b_j+c$ uses at most two one-bit additions:
\[
t_1=(a_j+b_j)\quad\text{and}\quad t_2=t_1+c.
\]
If $c=0$ and $a_j+b_j<2$, the second add is effectively a no-op, so the total count is \emph{strictly less than $2n$}. In any case, the bound $\le 2n$ yields linear time $O(n)$.

\section*{Worked examples}

\subsection*{A. Two-bit example: $(11)_2 + (01)_2$}
\begin{center}
\begin{tabular}{c|c c c|c c}
$j$ & $a_j$ & $b_j$ & $c$ (in) & $s_j$ & $c$ (out)\\\hline
0 & 1 & 1 & 0 & $0$ & $1$\\
1 & 1 & 0 & 1 & $0$ & $1$\\\hline
\multicolumn{6}{c}{$s_2=1$}
\end{tabular}
\end{center}
Result: $(100)_2$. Bit additions used $\le 4$.

\subsection*{B. Four-bit example: $(1011)_2 + (0110)_2$}
\begin{center}
\begin{tabular}{c|c c c|c c}
$j$ & $a_j$ & $b_j$ & $c$ (in) & $s_j$ & $c$ (out)\\\hline
0 & 1 & 0 & 0 & $1$ & $0$\\
1 & 1 & 1 & 0 & $0$ & $1$\\
2 & 0 & 1 & 1 & $0$ & $1$\\
3 & 1 & 0 & 1 & $0$ & $1$\\\hline
\multicolumn{6}{c}{$s_4=1$}
\end{tabular}
\end{center}
Result: $(10001)_2$. Bit additions used $\le 8$.

\section*{Solutions to student practice}

\subsection*{1) $(0101)_2 + (0011)_2$}
\begin{center}
\begin{tabular}{c|c c c|c c}
$j$ & $a_j$ & $b_j$ & $c$ (in) & $s_j$ & $c$ (out)\\\hline
0 & 1 & 1 & 0 & $0$ & $1$\\
1 & 0 & 1 & 1 & $0$ & $1$\\
2 & 1 & 0 & 1 & $0$ & $1$\\
3 & 0 & 0 & 1 & $1$ & $0$\\\hline
\multicolumn{6}{c}{$s_4=0$}
\end{tabular}
\end{center}
Answer: $(10000)_2$.

\subsection*{2) $(1001)_2 + (0001)_2$}
\begin{center}
\begin{tabular}{c|c c c|c c}
$j$ & $a_j$ & $b_j$ & $c$ (in) & $s_j$ & $c$ (out)\\\hline
0 & 1 & 1 & 0 & $0$ & $1$\\
1 & 0 & 0 & 1 & $1$ & $0$\\
2 & 0 & 0 & 0 & $0$ & $0$\\
3 & 1 & 0 & 0 & $1$ & $0$\\\hline
\multicolumn{6}{c}{$s_4=0$}
\end{tabular}
\end{center}
Answer: $(1010)_2$.

\subsection*{3) $(11101101)_2 + (10111011)_2$}
Work right-to-left; lots of carries chain through:
\begin{center}
\begin{tabular}{c|c c c|c c}
$j$ & $a_j$ & $b_j$ & $c$ (in) & $s_j$ & $c$ (out)\\\hline
0 & 1 & 1 & 0 & 0 & 1\\
1 & 0 & 1 & 1 & 0 & 1\\
2 & 1 & 1 & 1 & 1 & 1\\
3 & 1 & 0 & 1 & 0 & 1\\
4 & 0 & 1 & 1 & 0 & 1\\
5 & 1 & 1 & 1 & 1 & 1\\
6 & 1 & 0 & 1 & 0 & 1\\
7 & 1 & 1 & 1 & 1 & 1\\\hline
\multicolumn{6}{c}{$s_8=1$}
\end{tabular}
\end{center}
Answer: $(1\,011010000)_2=(1011010000)_2$.

\paragraph{Bit-addition count bound.} In all three problems, the number of one-bit additions is $<2n$ and $\le 2n$, so linear in the input length.

\end{document}

