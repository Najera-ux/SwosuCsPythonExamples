\documentclass[12pt]{article}
\usepackage[margin=1in]{geometry}
\usepackage{amsmath,amssymb}
\setlength{\parskip}{0.6em}
\setlength{\parindent}{0pt}

\begin{document}
{\large \textbf{Discrete Structures \quad Chapter 4.6 — Cryptography}}

\hrule
\vspace{0.6em}

\section*{Solutions — Example 5 Practice}

\subsection*{Problem A (easier)}
Ciphertext: \texttt{URYYB JBEYQ}; shift \(k=13\) (ROT13). \\
Decrypt by \(p \equiv c-13 \pmod{26}\) (or apply ROT13 again):
\[
\boxed{\texttt{HELLO WORLD}}
\]

\subsection*{Problem B (similar)}
Ciphertext: \texttt{ZHOFRPH WR FODVV}, unknown \(k\). \\
Guess that \texttt{WR} is \texttt{TO}. Then \(W=22\) should map to \(T=19\), so \(k\equiv 22-19\equiv 3\) and decryption uses \(p\equiv c-3\pmod{26}\). Check also that \texttt{FODVV} becomes \texttt{CLASS}:
\[
F(5)\to C(2),\ O(14)\to L(11),\ D(3)\to A(0),\ V(21)\to S(18),\ V(21)\to S(18).
\]
Hence \(k=3\) and
\[
\boxed{\texttt{WELCOME TO CLASS}}
\]

\subsection*{Problem C (harder)}
Ciphertext: \texttt{YMJ VZNHP GWTBS KTC OZRUX TAJW YMJ QFED ITL}. \\
The trigram \texttt{YMJ} repeats and often corresponds to \texttt{THE}. If so,
\[
Y(24) \to T(19) \Rightarrow k \equiv 24-19 \equiv 5,\quad \text{so decrypt with } p\equiv c-5\pmod{26}.
\]
Applying \(k=5\) across the text yields:
\[
\boxed{\texttt{THE QUICK BROWN FOX JUMPS OVER THE LAZY DOG}}
\]

\bigskip
\hrule
\bigskip

\textbf{Takeaways.}
\begin{itemize}
  \item Shift ciphers preserve frequency shape; a good guess (E, T, A, O) usually cracks \(k\).
  \item Decryption rule: \(p\equiv c-k\pmod{26}\); verify the guess by reading for sensible English.
  \item Longer texts make frequency clues stronger; short texts can be ambiguous, so test multiple hypotheses.
\end{itemize}

\end{document}

