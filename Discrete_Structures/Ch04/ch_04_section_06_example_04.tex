\documentclass[12pt]{article}
\usepackage[margin=1in]{geometry}
\usepackage{amsmath,amssymb}
\usepackage{enumitem}
\usepackage{setspace}
\setlength{\parskip}{0.6em}
\setlength{\parindent}{0pt}

\newcommand{\workbox}[1]{\fbox{\parbox{\dimexpr\textwidth-2\fboxsep-2\fboxrule\relax}{\vspace{#1}}}}

\begin{document}
{\large \textbf{Discrete Structures \quad Chapter 4.6 — Cryptography}}

\hrule
\vspace{0.6em}

\section*{Example 4 — Affine Cipher Warm-Up}

\textbf{Goal.} Determine which letter replaces \texttt{K} when the encryption function
\[
f(p) = (7p + 3) \bmod 26
\]
is used.

\subsection*{Big idea (the why):}
The affine cipher multiplies the plaintext value by a “stretch” factor and then shifts it.
It combines multiplication and addition inside modular arithmetic.

\[
\text{Encryption: } E(p) = (a p + b)\bmod 26
\qquad\quad
\text{Decryption: } D(c) = a^{-1}(c - b)\bmod 26.
\]
The constants \(a\) and \(b\) are keys.  \(a\) must be coprime to 26 so that \(a^{-1}\) exists.

\subsection*{Step 1 — Convert letter K to a number}
\[
K \rightarrow 10
\]

\subsection*{Step 2 — Apply the function \(f(p)=(7p+3)\bmod 26\)}
\[
f(10) = (7\cdot10 + 3)\bmod 26 = 73\bmod26 = 21.
\]

\subsection*{Step 3 — Convert number 21 back to a letter}
\[
21 \rightarrow V
\]

\textbf{Result:} \(K\) is encrypted as \(V\).

\subsection*{Why it works:}
Multiplying by 7 mixes up the order of letters more effectively than a simple shift, yet
because 7 and 26 are coprime, every letter still maps to exactly one output.

\bigskip
\hrule
\vspace{0.5em}

\section*{Practice (your turn!)}

\textbf{Problem A (easier).} Using \(f(p)=(3p+1)\bmod26\), find what letter replaces \texttt{C}.  
\emph{Hint:} \(C=2\).
\workbox{2cm}

\textbf{Problem B (similar).} Using \(f(p)=(5p+7)\bmod26\), find what letter replaces \texttt{H}.  
\emph{Hint:} compute carefully, mod 26.
\workbox{2cm}

\textbf{Problem C (harder).} Encrypt the word \texttt{DOG} using \(f(p)=(11p+8)\bmod26\).  
Write each step clearly: letter → number → formula → result → letter.
\workbox{3cm}

\textbf{Reflection.} Why must \(a\) be coprime with 26 for this cipher to be reversible?
\workbox{2cm}

\end{document}

