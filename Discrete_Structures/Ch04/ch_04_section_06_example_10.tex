\documentclass[12pt]{article}
\usepackage[margin=1in]{geometry}
\usepackage{amsmath, amssymb}
\usepackage{array, booktabs}
\usepackage{enumitem}
\usepackage{setspace}
\setlength{\parskip}{0.6em}
\setlength{\parindent}{0pt}

% roomy writing box for answers
\newcommand{\workbox}[1]{\fbox{\parbox{\dimexpr\textwidth-2\fboxsep-2\fboxrule\relax}{\vspace{#1}}}}

\begin{document}
{\large \textbf{Discrete Structures \quad Chapter 4.6 — Digital Signatures and RSA Authentication}}

\hrule
\vspace{0.8em}

\section*{Example 10 — Signing a Message with RSA}

\textbf{Scenario:}  
Alice’s public RSA cryptosystem uses \( n = 43 \times 59 = 2537 \) and \( e = 13 \).  
Her private key is \( d = 937 \), as computed in Example 9.  
She wishes to send the message “MEET AT NOON” to her friends so that they are \emph{certain} it came from her.

\vspace{0.5em}
\textbf{Goal:} Learn how RSA can be used to \emph{sign} a message—proving authenticity, not just secrecy.

\subsection*{Step 1 — Translate the message into numbers}
Using the standard letter-number system (A=00, B=01, …, Z=25):  
\[
M = \text{MEET AT NOON} \Rightarrow 1204\ 0419\ 0019\ 1314\ 1413
\]
(Verify this translation carefully! It’s essential to the encryption and decryption process.)

\subsection*{Step 2 — Apply Alice’s \emph{private key} to each block}
Alice uses her private key \( d = 937 \) to compute:
\[
x^{937} \pmod{2537}
\]
for each message block \( x \).  
This operation produces a “digital signature” that can only be generated with Alice’s private key.

\subsection*{Step 3 — Compute the results (with modular exponentiation)}
Using fast modular exponentiation (as in Example 9):
\[
\begin{aligned}
1204^{937} &\equiv 817 \pmod{2537} \\
0419^{937} &\equiv 555 \pmod{2537} \\
0019^{937} &\equiv 1310 \pmod{2537} \\
1314^{937} &\equiv 2173 \pmod{2537} \\
1413^{937} &\equiv 1026 \pmod{2537}
\end{aligned}
\]

So, the message Alice sends (in blocks) is:
\[
0817\ 0555\ 1310\ 2173\ 1026
\]

\subsection*{Step 4 — Verification by the recipient}
When her friends receive the message, they apply Alice’s \emph{public key} \( e = 13 \) to each block:
\[
E_{(2537,13)}(c) = c^{13} \pmod{2537}
\]
This reverses Alice’s signature and recovers the plaintext.  
If the recovered message reads “MEET AT NOON,” they know it truly came from Alice.

\subsection*{Key takeaway}
Digital signatures use the \emph{private key to sign} and the \emph{public key to verify}.  
This is the opposite direction from encryption (where public encrypts and private decrypts).  
It guarantees message authenticity and integrity — no one else could have produced this result.

\vspace{1em}
\hrule
\vspace{0.5em}

\section*{Practice — Your Turn!}

\textbf{Problem A (Warm-up):}  
Alice’s key is \( n = 77 \), \( e = 13 \), and \( d = 37 \).  
She wants to sign the message “HI,” represented as \( 0708 \).  
Compute the signature block \( c = m^d \pmod{77} \).  
Then, verify that \( c^e \pmod{77} \) returns \( 0708 \).
\workbox{3cm}

\textbf{Problem B (Moderate):}  
Bob uses the same RSA parameters as Example 10: \( n = 2537 \), \( e = 13 \), \( d = 937 \).  
He signs the message “HELP” (encoded as \( 0704\ 1115 \)).  
Compute \( m^d \pmod{2537} \) for each block, and verify correctness.
\workbox{4cm}

\textbf{Problem C (Challenge):}  
Suppose Eve intercepts Alice’s public key \( (2537, 13) \) and one of her signed messages.  
Why can’t Eve “fake” Alice’s signature without knowing \( d = 937 \)?  
Use your understanding of modular arithmetic and factorization to explain the barrier to forgery.  
\workbox{4cm}

\vspace{1em}
\textbf{Reflection:}  
Describe in your own words how digital signatures strengthen security compared to regular RSA encryption.  
\workbox{2cm}

\vspace{1em}
\hrule
\textbf{Quick Tips:}
\begin{itemize}[leftmargin=1.25em]
  \item Public key \( (n, e) \) — used to verify or encrypt.
  \item Private key \( d \) — used to sign or decrypt.
  \item Large primes \( p, q \) make \( n = pq \) hard to factor, ensuring security.
  \item Modular arithmetic is your shield: it keeps numbers within manageable limits.
\end{itemize}

\end{document}

