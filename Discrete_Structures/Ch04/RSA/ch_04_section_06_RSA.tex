\documentclass[12pt]{article}
\usepackage{amsmath, amssymb, fancyhdr, geometry}
\geometry{margin=1in}
\pagestyle{fancy}
\fancyhead[L]{\textbf{Discrete Structures}}
\fancyhead[C]{Chapter 4.6 --- RSA Encryption}
\fancyhead[R]{\textbf{Example: Simple RSA by Hand}}

\begin{document}

\begin{center}
    \Large \textbf{Example: A Simple RSA Demonstration}\\[1em]
    \normalsize (Toy model — not secure, but perfect for learning!)
\end{center}

\section*{Goal}
Encrypt and decrypt the message \( M = 7 \) using a tiny RSA setup.

\section*{Step 1. Choose primes}
Let \( p = 5 \) and \( q = 11 \).\\[0.5em]
Then
\[
n = p \times q = 5 \times 11 = 55,
\quad
\phi(n) = (p - 1)(q - 1) = 4 \times 10 = 40.
\]

\section*{Step 2. Choose public key exponent \( e \)}
We need \( e \) such that \( 1 < e < 40 \) and \( \gcd(e, 40) = 1 \).\\
Let’s choose \( e = 3 \), since \(\gcd(3,40) = 1\).

\section*{Step 3. Compute private key exponent \( d \)}
We need \( d \) such that
\[
e \times d \equiv 1 \pmod{40}.
\]
Try small values:
\[
3 \times 27 = 81 \equiv 1 \pmod{40}.
\]
So \( d = 27 \).

\[
\textbf{Public key: } (n, e) = (55, 3)
\quad\quad
\textbf{Private key: } (n, d) = (55, 27)
\]

\section*{Step 4. Encrypt a message}
Let our message be \( M = 7 \).  
Compute ciphertext:
\[
C \equiv M^e \pmod{n} = 7^3 \bmod 55 = 343 \bmod 55 = 13.
\]

\[
\boxed{C = 13}
\]

\section*{Step 5. Decrypt the ciphertext}
Now compute:
\[
M \equiv C^d \pmod{n} = 13^{27} \bmod 55.
\]
We can reduce step-by-step (or use a calculator):

\[
13^2 \equiv 4, \quad 13^4 \equiv 16, \quad 13^8 \equiv 36, \quad 13^{16} \equiv 31,
\]
and after combining exponents properly,
\[
13^{27} \bmod 55 = 7.
\]

\[
\boxed{M = 7 \text{ (original message recovered!)}}
\]

\section*{Summary}
\begin{itemize}
    \item \( p=5, q=11 \Rightarrow n=55, \phi=40 \)
    \item \( e=3, d=27 \)
    \item Encrypt \( M=7 \Rightarrow C=13 \)
    \item Decrypt \( C=13 \Rightarrow M=7 \)
\end{itemize}

\noindent Even though our numbers are tiny, this is exactly the same math that powers real RSA with 2048-bit primes.

\end{document}

