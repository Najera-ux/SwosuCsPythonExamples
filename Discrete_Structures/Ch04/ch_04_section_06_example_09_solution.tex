\documentclass[12pt]{article}
\usepackage[margin=1in]{geometry}
\usepackage{amsmath,amssymb,booktabs,array}
\setlength{\parskip}{0.65em}
\setlength{\parindent}{0pt}

\begin{document}
{\large \textbf{Discrete Structures \qquad Chapter 4.6 — Cryptography}}\\
\textbf{Solutions: Example 9 (RSA Decryption)}

\hrule
\vspace{0.6em}

\section*{Core facts for this example}
Public key: \((n,e)=(2537,13)\) with \(n=43\cdot59\).\\
Euler totient: \(\phi(n)=(p-1)(q-1)=42\cdot58=2436\).\\
Private exponent \(d\) is the inverse of \(e\) modulo \(\phi(n)\):
\[
13d\equiv 1\pmod{2436}\quad\Rightarrow\quad d=937.
\]
Decryption works blockwise: for each ciphertext block \(c\),
\[
m \equiv c^{\,d}\pmod{n} \qquad\text{and then map } m\text{ back to letters with }A=00,\dots,Z=25.
\]

\section*{Textbook Example 9 — Full decryption}
Ciphertext: \(\texttt{0981 0461}\).

\subsection*{Block 1: \(c=0981\Rightarrow 981\)}
We use repeated squaring (mod \(2537\)) and write \(d=937=512+256+128+32+8+1\).
\[
\begin{array}{r|rrrrrrrrr}
\text{power} & 1 & 2 & 4 & 8 & 16 & 32 & 64 & 128 & 256 & 512\\\hline
981^{\text{power}} \bmod 2537 &
981 & 838 & 1922 & 1325 & 450 & 441 & 322 & 2472 & 1688 & 293
\end{array}
\]
Multiply only the needed entries (powers \(1,8,32,128,256,512\)), reducing after each step:
\[
981\cdot1325\cdot441\cdot2472\cdot1688\cdot293 \equiv \boxed{704}\pmod{2537}.
\]
So \(m_1=0704\Rightarrow \texttt{07 04}=\texttt{H E}\).

\subsection*{Block 2: \(c=0461\Rightarrow 461\)}
Again with \(d=937=512+256+128+32+8+1\):
\[
\begin{array}{r|rrrrrrrrr}
\text{power} & 1 & 2 & 4 & 8 & 16 & 32 & 64 & 128 & 256 & 512\\\hline
461^{\text{power}} \bmod 2537 &
461 & 1950 & 2074 & 1261 & 1959 & 1737 & 676 & 316 & 913 & 1433
\end{array}
\]
Multiply the needed entries (powers \(1,8,32,128,256,512\)):
\[
461\cdot1261\cdot1737\cdot316\cdot913\cdot1433 \equiv \boxed{1115}\pmod{2537}.
\]
So \(m_2=1115\Rightarrow \texttt{11 15}=\texttt{L P}\).

\textbf{Plaintext:} \(\boxed{\texttt{HELP}}\).

\medskip
\textit{Checks and teaching notes.} Emphasize (i) mapping is two digits per letter with leading zeros preserved; (ii) block size \(4\) works because \(2525<n=2537<252525\); (iii) reduce after \emph{every} multiplication to keep numbers small.

\hrule
\vspace{0.5em}

\section*{Practice Solutions}

\subsection*{Problem A (easier)}
\textbf{Prompt.} Decrypt the single block \(\texttt{2081}\).

\textbf{Work.} Compute \(m\equiv 2081^{937}\pmod{2537}\). (Repeated squaring or any correct modular-pow tool is fine.) One clean path gives
\[
2081^{937}\equiv \boxed{1819}\pmod{2537}.
\]
Split to letters: \(18\to \texttt{S}\), \(19\to \texttt{T}\). \\
\textbf{Answer:} \(\boxed{\texttt{ST}}\).

\subsection*{Problem B (similar)}
\textbf{Prompt.} Decrypt the two blocks \(\texttt{2081 2182}\).

\textbf{Work.} From part A, \(2081^{937}\equiv 1819\Rightarrow \texttt{ST}\).
Similarly,
\[
2182^{937}\equiv \boxed{1415}\pmod{2537}\Rightarrow \texttt{14 15}=\texttt{O P}.
\]
\textbf{Answer:} \(\boxed{\texttt{STOP}}\).

\subsection*{Problem C (harder)}
\textbf{Prompt.} Decrypt \(\texttt{0981 0724 1774}\). Same key.

\textbf{Work.} Blockwise decryption:
\[
\begin{aligned}
0981^{937}&\equiv \boxed{0704}\pmod{2537} &&\Rightarrow \texttt{HE},\\
0724^{937}&\equiv \boxed{1111}\pmod{2537} &&\Rightarrow \texttt{LL},\\
1774^{937}&\equiv \boxed{1423}\pmod{2537} &&\Rightarrow \texttt{OX}.
\end{aligned}
\]
\textbf{Answer:} \(\boxed{\texttt{HELLOX}}\) (final \texttt{X} is padding to complete a two-letter block).

\medskip
\textit{Coach’s notes.}
\begin{itemize}
\item When a message length is odd (in letters), a padding letter (commonly \texttt{X}) is appended so every numeric string splits cleanly into four-digit blocks.
\item If students’ intermediate residues differ, check two things: (1) their exponent decomposition of \(937\) and (2) that they reduced modulo \(2537\) after \emph{every} multiply and square.
\end{itemize}

\hrule
\vspace{0.4em}
\textbf{Quick reference: letter map (A=00,\dots,Z=25).}\\
\(\{00,01,\dots,09\}\to\{\texttt{A},\texttt{B},\dots,\texttt{J}\},\;
10\to\texttt{K},\;11\to\texttt{L},\;\dots,\;25\to\texttt{Z}.\)

\end{document}

