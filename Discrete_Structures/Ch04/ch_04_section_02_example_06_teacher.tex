\documentclass[12pt]{article}
\usepackage[margin=1in]{geometry}
\usepackage{amsmath,amssymb}

\begin{document}

\begin{center}
\Large\textbf{Solutions: Binary Expansion (Example 6)}
\end{center}

\section*{Part A — Worked Example}
\[
241 \div 2 = 120 \, r1
\]
\[
120 \div 2 = 60 \, r0
\]
\[
60 \div 2 = 30 \, r0
\]
\[
30 \div 2 = 15 \, r0
\]
\[
15 \div 2 = 7 \, r1
\]
\[
7 \div 2 = 3 \, r1
\]
\[
3 \div 2 = 1 \, r1
\]
\[
1 \div 2 = 0 \, r1
\]

Reading bottom-to-top: $(11110001)_{2}$.

---

\section*{Part B — Easier Practice Solutions}

1. $(13)_{10}$:  
\[
13 \div 2 = 6 \, r1,\;
6 \div 2 = 3 \, r0,\;
3 \div 2 = 1 \, r1,\;
1 \div 2 = 0 \, r1.
\]  
Answer: $(13)_{10} = (1101)_{2}$.

\medskip
2. $(100)_{10}$:  
\[
100 \div 2 = 50 \, r0,\;
50 \div 2 = 25 \, r0,\;
25 \div 2 = 12 \, r1,\;
12 \div 2 = 6 \, r0,\;
6 \div 2 = 3 \, r0,\;
3 \div 2 = 1 \, r1,\;
1 \div 2 = 0 \, r1.
\]  
Answer: $(100)_{10} = (1100100)_{2}$.

---

\section*{Part C — Harder Challenge Solution}

$(1023)_{10}$. Note: $1023 = 2^{10} - 1$.

This means the binary expansion will be ten 1's in a row.

\[
(1023)_{10} = (1111111111)_{2}.
\]

---

\section*{Teaching Notes}
\begin{itemize}
  \item Reinforce the bottom-to-top reading of remainders.
  \item Use powers of $2$ to recognize special forms (like $2^n - 1$).
  \item Encourage students to double-check by recomputing in decimal.
\end{itemize}

\end{document}

