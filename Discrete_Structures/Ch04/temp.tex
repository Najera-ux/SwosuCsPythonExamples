\documentclass[12pt]{article}
\usepackage[margin=1in]{geometry}
\usepackage{amsmath,amssymb}
\usepackage{array,tabularx,booktabs}
\usepackage{enumitem}
\usepackage{setspace}
\setlength{\parskip}{0.6em}
\setlength{\parindent}{0pt}

% roomy writing space boxes
\newcommand{\workbox}[1]{\fbox{\parbox{\dimexpr\textwidth-2\fboxsep-2\fboxrule\relax}{\vspace{#1}}}}

\begin{document}
{\large \textbf{Discrete Structures \quad Chapter 4.6 — Cryptography}}

\hrule
\vspace{0.6em}

\section*{Example 2 (Worksheet) — Shift Cipher with \(k=11\)}

\textbf{Goal.} Encrypt the message \texttt{STOP GLOBAL WARMING} using Caesar's shift cipher with \(k=11\).

\subsection*{Big idea (the ``why''):}%
We model letters as numbers in \(\mathbb{Z}_{26}\) so that a shift is just \emph{modular addition}. This keeps us in the alphabet and gives the wrap-around from \(Z\) back to \(A\).

\[
\text{A}=0,\ \text{B}=1,\ \ldots,\ \text{Z}=25
\qquad\qquad
E_k(p) = (p + k) \bmod 26.
\]

For this example, \(k=11\).

\subsection*{Step 1 — Normalize and map letters \(\to\) numbers}
We use uppercase and keep spaces. Convert each letter of \texttt{STOP GLOBAL WARMING} to its number:
\[
\underbrace{\texttt{STOP}}_{18\;19\;14\;15}
\quad
\underbrace{\texttt{GLOBAL}}_{6\;11\;14\;1\;0\;11}
\quad
\underbrace{\texttt{WARMING}}_{22\;0\;17\;12\;8\;13\;6}.
\]

\subsection*{Step 2 — Apply the shift \(k=11\) (add 11 mod 26)}
Compute \(c \equiv p+11 \pmod{26}\) for each number. Do the wrap when you go past 25.
\[
\begin{aligned}
\texttt{STOP}:&\quad 18,19,14,15 \mapsto 3,4,25,0 \\
\texttt{GLOBAL}:&\quad 6,11,14,1,0,11 \mapsto 17,22,25,12,11,22 \\
\texttt{WARMING}:&\quad 22,0,17,12,8,13,6 \mapsto 7,11,2,23,19,24,17.
\end{aligned}
\]

\subsection*{Step 3 — Map numbers \(\to\) letters and keep spaces}
\[
3,4,25,0\ \ |\ 17,22,25,12,11,22\ \ |\ 7,11,2,23,19,24,17
\quad\Rightarrow\quad
\boxed{\texttt{DEZA RWZMLW HLCXTYR}}
\]

\subsection*{Helpful tips \& common pitfalls}
\begin{itemize}[leftmargin=1.25em]
  \item \textbf{A=0, not 1.} Off-by-one mistakes are the #1 bug.
  \item \textbf{Wrap cleanly:} if \(p+k\ge 26\), subtract 26 (i.e., reduce mod 26).
  \item \textbf{Spaces/punctuation} pass through unchanged; only letters get shifted.
  \item \textbf{Decrypting} with \(k=11\) is the same as adding \(-11\) (or \(+15\)) mod 26.
\end{itemize}

\bigskip
\hrule
\vspace{0.6em}

\section*{Practice (your turn!)}

\textbf{Problem A (easier).} Encrypt with \(k=4\): \texttt{MATH IS FUN} \\
\emph{Why:} smaller shift, shorter phrase—perfect confidence builder.
\workbox{2.0cm}

\textbf{Problem B (similar).} Decrypt with \(k=11\): \texttt{SPWWZ HZCWO} \\
\emph{Tip:} subtract 11 mod 26 or add 15.
\workbox{2.0cm}

\textbf{Problem C (harder).} Unknown \(k\). Decrypt the Caesar ciphertext: \texttt{P HT HA AOL WHYR} \\
\emph{Hints:} a one-letter word is often \texttt{I} or \texttt{A}. The block \texttt{AOL} frequently shows up when ``\texttt{THE}'' is encrypted with \(k=7\).
\workbox{2.4cm}

\bigskip
\textbf{Reflection.} In one sentence: explain why modular arithmetic guarantees a valid letter after every shift. \\
\workbox{1.2cm}

\end{document}

