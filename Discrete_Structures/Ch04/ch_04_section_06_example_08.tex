\documentclass[12pt]{article}
\usepackage[margin=1in]{geometry}
\usepackage{amsmath,amssymb}
\usepackage{array,booktabs,tabularx}
\usepackage{enumitem}
\usepackage{setspace}
\setlength{\parskip}{0.6em}
\setlength{\parindent}{0pt}

% Define a roomy workspace box
\newcommand{\workbox}[1]{\fbox{\parbox{\dimexpr\textwidth-2\fboxsep-2\fboxrule\relax}{\vspace{#1}}}}

\begin{document}
{\large \textbf{Discrete Structures \quad Chapter 4.6 — RSA Encryption}}

\hrule
\vspace{0.6em}

\section*{Example 8 (Worksheet) — Encrypting with the RSA Cryptosystem}

\textbf{Goal.} Encrypt the message \texttt{STOP} using the RSA cryptosystem with key \((n,e) = (2537, 13)\).

\subsection*{Background idea}
RSA is a \textbf{public key cryptosystem}. Anyone can use the public key \((n,e)\) to encrypt, but only the private key (involving \(d\)) can decrypt.  
Each letter is first turned into a number (A=00, B=01, ..., Z=25), grouped into blocks that fit under \(n\), and then encrypted using
\[
c \equiv m^e \pmod{n}.
\]

\subsection*{Step 1 — Convert letters to numbers}
We map \texttt{STOP} as:
\[
\texttt{S T O P} \Rightarrow 18\ 19\ 14\ 15.
\]
Group into four-digit blocks:
\[
1819 \quad 1415
\]
(because \(2525 < 2537 < 252525\), so 4 digits per block fits safely).

\subsection*{Step 2 — Apply RSA encryption}
For each block \(m\), compute
\[
c \equiv m^{13} \pmod{2537}.
\]
You can use fast modular exponentiation (successive squaring) to simplify:
\[
1819^{13} \pmod{2537} = 2081, \qquad 1415^{13} \pmod{2537} = 2182.
\]
Hence, the ciphertext is:
\[
\boxed{\texttt{2081 2182}}.
\]

\subsection*{Step 3 — Interpretation}
We transmit \texttt{2081 2182}. Only someone with the private key \(d\) (that satisfies \(ed \equiv 1 \pmod{(p-1)(q-1)}\)) can decrypt the message.

\subsection*{Tips \& tricks}
\begin{itemize}[leftmargin=1.25em]
  \item \textbf{Why 13?} — Because \(\gcd(13, (p-1)(q-1)) = 1\), ensuring encryption is reversible.
  \item \textbf{Always check block size.} \(m\) must be smaller than \(n\).
  \item \textbf{Decryption uses the inverse of \(e\)} — It “undoes” the exponentiation by modular arithmetic symmetry.
  \item \textbf{RSA loves primes.} Choosing \(p,q\) large keeps \(n\) hard to factor.
\end{itemize}

\bigskip
\hrule
\vspace{0.6em}

\section*{Practice — Your Turn}

\textbf{Problem A (easier).} Encrypt \texttt{GO} using RSA with \((n,e)=(2537,13)\).  
Hint: Convert \texttt{GO} → 06014 → use 4-digit block \(0601\), compute \(c \equiv m^{13} \pmod{2537}\).
\workbox{2.2cm}

\textbf{Problem B (similar).} Encrypt \texttt{HELP} using RSA with \((n,e)=(2537,13)\).  
Show all modular exponentiation steps clearly.
\workbox{3.0cm}

\textbf{Problem C (challenge).} Encrypt \texttt{SAVE THE PLANET} using RSA with \((n,e)=(2537,13)\).  
Break your message into 4-digit blocks and compute each ciphertext block.  
(Hint: spaces can be ignored or replaced by 26.)
\workbox{4.2cm}

\bigskip
\textbf{Reflection.} In one or two sentences, explain \emph{why} RSA’s security depends on factoring large primes.
\workbox{1.5cm}

\end{document}

