\documentclass[12pt]{article}
\usepackage[margin=1in]{geometry}
\usepackage{amsmath,amssymb}
\usepackage{array,booktabs,tabularx}
\usepackage{enumitem}
\usepackage{setspace}
\setlength{\parskip}{0.6em}
\setlength{\parindent}{0pt}

% roomy writing space boxes
\newcommand{\workbox}[1]{\fbox{\parbox{\dimexpr\textwidth-2\fboxsep-2\fboxrule\relax}{\vspace{#1}}}}

\begin{document}
{\large \textbf{Discrete Structures \quad Chapter 4.6 — Cryptography}}

\hrule
\vspace{0.6em}

\section*{Example 6 (Worksheet) — Transposition Cipher with a Permutation}

\textbf{Cipher rule (why it’s cool).}  
A \emph{transposition} cipher keeps the letters but shuffles their \emph{positions}.  
We split plaintext into blocks of 4 and apply the permutation
\[
\sigma = \begin{bmatrix}1&2&3&4\\ 3&1&4&2\end{bmatrix}.
\]
That is: \underline{1st}\(\to\)3rd,\ \underline{2nd}\(\to\)1st,\ \underline{3rd}\(\to\)4th,\ \underline{4th}\(\to\)2nd.  
(So for plaintext block \(p_1p_2p_3p_4\) the ciphertext block is \(c_1c_2c_3c_4 = p_2\,p_4\,p_1\,p_3\).)

\subsection*{(a) Encrypt \texttt{PIRATE ATTACK}}
\textbf{Step 1 — Normalize and block (remove spaces, then group 4).}
\[
\texttt{PIRATEATTACK} \ \Rightarrow\ \texttt{PIRA}\ \texttt{TEAT}\ \texttt{TACK}.
\]

\textbf{Step 2 — Apply }\(\sigma\) \textbf{to each block.}
\[
\begin{aligned}
\texttt{PIRA}&:\ p_1\!=\!P,\ p_2\!=\!I,\ p_3\!=\!R,\ p_4\!=\!A \Rightarrow c = p_2\,p_4\,p_1\,p_3 = \texttt{IAPR},\\
\texttt{TEAT}&:\ p=\texttt{T,E,A,T} \Rightarrow c=\texttt{E T T A},\\
\texttt{TACK}&:\ p=\texttt{T,A,C,K} \Rightarrow c=\texttt{A K T C}.
\end{aligned}
\]
\textbf{Ciphertext:} \(\boxed{\texttt{IAPR ETTA AKTC}}\).

\subsection*{(b) Decrypt \texttt{SWUE TRAE OEHS}}
To undo the shuffle, use \(\sigma^{-1}\):
\[
\sigma^{-1}=\begin{bmatrix}1&2&3&4\\ 2&4&1&3\end{bmatrix}
\quad(\text{so } c_1\!\to\!p_2,\ c_2\!\to\!p_4,\ c_3\!\to\!p_1,\ c_4\!\to\!p_3).
\]
\textbf{Block and apply \(\sigma^{-1}\):}
\[
\texttt{SWUE}\!\to\!\texttt{USEW},\qquad
\texttt{TRAE}\!\to\!\texttt{ATER},\qquad
\texttt{OEHS}\!\to\!\texttt{HOSE}.
\]
\textbf{Plaintext (grouped):} \(\boxed{\texttt{USE WATER HOSE}}\).

\subsection*{Tips \& pitfalls}
\begin{itemize}[leftmargin=1.25em]
  \item \textbf{Always block first.} Remove spaces, then group in 4s. If the last block is short, pad (e.g., with \texttt{X}).
  \item \textbf{Keep “from” vs “to” straight:} here \(\sigma\) says where each \emph{plaintext position} lands in ciphertext.
  \item \textbf{Decrypt with \(\sigma^{-1}\):} move each ciphertext position back to the correct plaintext spot.
\end{itemize}

\bigskip
\hrule
\vspace{0.5em}

\section*{Practice — Your Turn}

\textbf{Use the same permutation } \(\sigma=[3,1,4,2]\). Work neatly: show the block, show \(p_1p_2p_3p_4\), then the rearranged \(c_1c_2c_3c_4\).

\textbf{Problem A (easier).} Encrypt \texttt{MATH NERD}. (No padding needed.)
\workbox{2.2cm}

\textbf{Problem B (similar).} Decrypt the ciphertext \texttt{OEHM OKWR}.
\workbox{2.2cm}

\textbf{Problem C (harder).} Encrypt \texttt{DATA SCIENCE}. If needed, \emph{pad the last block with X} to fill 4 letters. Show every block and the final ciphertext.
\workbox{3.0cm}

\bigskip
\textbf{Reflection.} Why does transposition preserve letter frequencies but still hide the message structure?
\workbox{1.4cm}

\end{document}

