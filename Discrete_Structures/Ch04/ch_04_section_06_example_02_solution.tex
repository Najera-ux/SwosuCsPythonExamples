\documentclass[12pt]{article}
\usepackage[margin=1in]{geometry}
\usepackage{amsmath,amssymb}
\usepackage{array,booktabs}
\usepackage{enumitem}
\setlength{\parskip}{0.6em}
\setlength{\parindent}{0pt}

\begin{document}
{\large \textbf{Discrete Structures \quad Chapter 4.6 — Cryptography}}

\hrule
\vspace{0.6em}

\section*{Solutions for Example 2 Practice}

\subsection*{Problem A (easier). \quad Encrypt with \(k=4\): \texttt{MATH IS FUN}}
\textbf{Map to numbers (A=0):}
\[
\texttt{MATH IS FUN} \Rightarrow 12,0,19,7,\; 8,18,\; 5,20,13.
\]
\textbf{Add \(4\) mod 26:}
\[
12,0,19,7 \mapsto 16,4,23,11 \quad\ (\text{M}\to\text{Q},\ \text{A}\to\text{E},\ \ldots)
\]
\[
8,18 \mapsto 12,22 \qquad 5,20,13 \mapsto 9,24,17.
\]
\textbf{Back to letters:}
\[
16,4,23,11, \ 12,22, \ 9,24,17 \Rightarrow \boxed{\texttt{QEXL MW JYR}}.
\]
\emph{Why it works:} Every step is addition in \(\mathbb{Z}_{26}\); wrap ensures letters stay in 0–25.

\bigskip

\subsection*{Problem B (similar). \quad Decrypt with \(k=11\): \texttt{SPWWZ HZCWO}}
\textbf{Numbers for ciphertext:}
\[
\texttt{SPWWZ HZCWO} \Rightarrow 18,15,22,22,25,\; 7,25,2,22,14.
\]
\textbf{Subtract 11 (or add 15) mod 26:}
\[
18,15,22,22,25 \mapsto 7,4,11,11,14 \quad (\text{H,E,L,L,O})
\]
\[
7,25,2,22,14 \mapsto 22,14,17,11,3 \quad (\text{W,O,R,L,D}).
\]
\textbf{Plaintext:} \(\boxed{\texttt{HELLO WORLD}}\).

\bigskip

\subsection*{Problem C (harder). \quad Unknown \(k\): \texttt{P HT HA AOL WHYR}}
\textbf{Strategy (the why):} Look for patterns. A one-letter word is probably \texttt{I} or \texttt{A}. Also, \texttt{AOL} famously appears when ``\texttt{THE}'' is shifted by \(k=7\) (since \(19{+}7=26\equiv 0=\text{A}\), etc.).

\textbf{Infer \(k\):} If \(\texttt{AOL}\) is \(\texttt{THE}\), then the shift is \(k=7\).

\textbf{Decrypt by subtracting 7:}
\[
\texttt{P} \mapsto \texttt{I},\quad
\texttt{HT} \mapsto \texttt{AM},\quad
\texttt{HA} \mapsto \texttt{AT},\quad
\texttt{AOL} \mapsto \texttt{THE},\quad
\texttt{WHYR} \mapsto \texttt{PARK}.
\]
\[
\Rightarrow \boxed{\texttt{I AM AT THE PARK}}.
\]

\bigskip
\hrule
\bigskip

\textbf{Key takeaways.}
\begin{itemize}[leftmargin=1.25em]
  \item Encryption: \(E_k(p)=(p+k)\bmod 26\), Decryption: \(D_k(c)=(c-k)\bmod 26\).
  \item Unknown \(k\) can be cracked with educated guesses (``\texttt{THE}'', one-letter words) or brute force (only 26 options).
  \item Thinking in \(\mathbb{Z}_{26}\) explains the wrap-around and keeps errors low.
\end{itemize}

\end{document}

