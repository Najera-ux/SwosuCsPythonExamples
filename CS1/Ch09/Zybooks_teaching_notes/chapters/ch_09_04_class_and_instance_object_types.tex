\chapter{Class and Instance Object Types}

\section*{9.4.1 Understanding Class vs. Instance Objects}

A class in Python acts as a \textbf{factory} that creates \textbf{instance objects}. 
Each instance has its own data (attributes), but shares the same methods defined in the class.

\begin{verbatim}
class Time:
    def __init__(self):
        self.hours = 0
        self.minutes = 0

time1 = Time()
time2 = Time()
time1.hours = 5
time2.hours = 7
\end{verbatim}

---

\section*{9.4.2 Class Attributes vs. Instance Attributes}

A \textbf{class attribute} is shared by all instances, while an \textbf{instance attribute} is unique to each object.

\begin{verbatim}
class MarathonRunner:
    race_distance = 42.195  # Class attribute

    def __init__(self):
        self.speed = 0      # Instance attribute

runner1 = MarathonRunner()
runner2 = MarathonRunner()
runner1.speed = 7.5
runner2.speed = 3.0
print(f"Runner1: {runner1.speed}, Runner2: {runner2.speed}")
\end{verbatim}


---

\section*{9.4.3 Key Concept Summary}

\begin{itemize}
    \item \textbf{Class Object:} A template that defines data and behavior.
    \item \textbf{Instance Object:} A unique copy created from the class.
    \item \textbf{Class Attribute:} Shared by all instances.
    \item \textbf{Instance Attribute:} Belongs only to one instance.
\end{itemize}

---

\section*{9.4.4 Practice Activity}

Modify the code below to add a method \texttt{print\_attributes()} 
that prints both the class attribute and the instance attribute values.

\begin{verbatim}
class PhoneNumber:
    area_code = "405"

    def __init__(self):
        self.number = "555-1234"

# TODO: Add print_attributes method
\end{verbatim}

\begin{center}
\textit{What happens if you change the class attribute after creating multiple instances?}
\end{center}

