\documentclass[12pt]{article}
\usepackage[utf8]{inputenc}
\usepackage[a4paper,margin=1in]{geometry}
\usepackage{fancyhdr}
\usepackage{titlesec}
\usepackage{amsmath,amssymb}
\usepackage{xcolor}
\usepackage{listings}
\usepackage{tcolorbox}
\tcbuselibrary{breakable,skins}
\usepackage{enumitem}
\usepackage{hyperref}
\usepackage{caption}

% --------------------------------------------------------------
% Page style
% --------------------------------------------------------------
\pagestyle{fancy}
\fancyhf{}
\rhead{Chapter 11.3: Importing Specific Names (Teacher Edition)}
\lhead{SWOSU Computer Science}
\rfoot{\thepage}
\setlength{\headheight}{15pt}

% --------------------------------------------------------------
% Listings setup
% --------------------------------------------------------------
\definecolor{codegray}{gray}{0.95}
\lstset{
  backgroundcolor=\color{codegray},
  basicstyle=\ttfamily\small,
  frame=single,
  breaklines=true,
  showstringspaces=false,
  keywordstyle=\color{blue},
  commentstyle=\color{gray},
  stringstyle=\color{purple},
  tabsize=4
}

% --------------------------------------------------------------
% Box definitions
% --------------------------------------------------------------
\newtcolorbox{activitybox}[2][]{colback=blue!5!white,colframe=blue!75!black,
  title={#2},fonttitle=\bfseries,#1,breakable}
\newtcolorbox{conceptbox}[2][]{colback=green!5!white,colframe=green!50!black,
  title={#2},fonttitle=\bfseries,#1,breakable}
\newtcolorbox{reflectionbox}[2][]{colback=orange!5!white,colframe=orange!60!black,
  title={#2},fonttitle=\bfseries,#1,breakable}
\newtcolorbox{teacherbox}[2][]{colback=yellow!10!white,colframe=yellow!60!black,
  title={Instructor Notes – #2},fonttitle=\bfseries\itshape,#1,breakable}

% --------------------------------------------------------------
\begin{document}

\begin{center}
  \vspace*{1cm}
  {\Huge \textbf{Chapter 11.3 – Importing Specific Names from a Module}}\\[0.5cm]
  {\Large Teacher Edition with Full Solutions}\\[1cm]
  \rule{\textwidth}{0.4pt}\\[1cm]
\end{center}

% ==============================================================
\section{Learning Objectives}
\begin{itemize}
  \item Import only specific functions or variables from a module.
  \item Use aliases to simplify long module names.
  \item Understand the benefits and risks of using the wildcard \texttt{*} import.
\end{itemize}

% ==============================================================
\section{Introduction}
Sometimes you don’t want to import an entire module—just one or two functions.  
Python allows you to import specific names directly using the syntax:
\begin{lstlisting}[language=Python]
from module_name import function_name
\end{lstlisting}

You can also import multiple items:
\begin{lstlisting}[language=Python]
from module_name import func1, func2
\end{lstlisting}

And for convenience:
\begin{lstlisting}[language=Python]
from module_name import function_name as shortname
\end{lstlisting}

---

\begin{conceptbox}{Example – Using math Functions}
\begin{lstlisting}[language=Python]
from math import sqrt, pi

print("Square root of 16:", sqrt(16))
print("Area of circle radius 2:", pi * (2 ** 2))
\end{lstlisting}

\textbf{Output:}
\begin{lstlisting}
Square root of 16: 4.0
Area of circle radius 2: 12.566370614359172
\end{lstlisting}
\end{conceptbox}

\begin{teacherbox}{Key Point}
Encourage students to recognize readability trade-offs:  
Explicit imports make code more readable, but too many can clutter the namespace.  
Contrast this with importing the full module and using prefixes.
\end{teacherbox}

% ==============================================================
\section{Using Aliases}

You can rename imported items or modules for convenience:

\begin{lstlisting}[language=Python]
import math as m

print(m.sqrt(81))
\end{lstlisting}

or:

\begin{lstlisting}[language=Python]
from math import factorial as f
print(f(5))
\end{lstlisting}

\begin{teacherbox}{Teaching Emphasis}
Aliases improve readability when module names are long, like \texttt{numpy} → \texttt{np}.  
Have students reflect on readability vs clarity.
\end{teacherbox}

% ==============================================================
\section{Wildcard Imports – Use With Caution}
You can import everything with:
\begin{lstlisting}[language=Python]
from math import *
\end{lstlisting}

This loads all public functions and variables into your current namespace.  
However, it makes it hard to tell where names came from and can cause conflicts.

\begin{activitybox}{Activity – Namespace Chaos}
Try running this experiment:

\begin{lstlisting}[language=Python]
from math import *
from random import *

print(sin(0))    # math.sin
print(random())  # random.random
\end{lstlisting}

Now define your own \texttt{sin()} function below and see what happens!
\begin{lstlisting}[language=Python]
def sin(x):
    return "This is not math.sin!"
print(sin(0))
\end{lstlisting}

\textbf{Expected Output:}
\begin{lstlisting}
0.0
0.3748298023
This is not math.sin!
\end{lstlisting}
\end{activitybox}

\begin{teacherbox}{Discussion Prompt}
Ask: “Why is this dangerous?”  
Students will see how one small override can break math functions silently.  
Encourage the use of explicit imports or aliases instead.
\end{teacherbox}

% ==============================================================
\section{Mini Challenge – Custom Utility Module}

\begin{activitybox}{Student Challenge}
\textbf{Step 1:} Create a module named \texttt{converter.py}:

\begin{lstlisting}[language=Python]
def c_to_f(c):
    return (c * 9/5) + 32

def f_to_c(f):
    return (f - 32) * 5/9
\end{lstlisting}

\textbf{Step 2:} In another file \texttt{main.py}, import specific functions:

\begin{lstlisting}[language=Python]
from converter import c_to_f, f_to_c

print("0°C =", c_to_f(0), "°F")
print("212°F =", f_to_c(212), "°C")
\end{lstlisting}

\textbf{Expected Output:}
\begin{lstlisting}
0°C = 32.0 °F
212°F = 100.0 °C
\end{lstlisting}
\end{activitybox}

\begin{teacherbox}{Instructor Solution Notes}
Show students that importing specific functions keeps the code lightweight.  
Contrast this with importing the whole module and calling \texttt{converter.c\_to\_f(0)}.
\end{teacherbox}

% ==============================================================
\section{Reflection – Why This Matters}

\begin{reflectionbox}{Core Takeaways}
\begin{itemize}
  \item Specific imports keep code concise and readable.
  \item Wildcard imports can create confusion and bugs.
  \item Aliases simplify long names and improve code style.
\end{itemize}
\end{reflectionbox}

\begin{teacherbox}{Wrap-Up Discussion}
Ask students to compare the following two lines:
\begin{lstlisting}
from math import sqrt
import math
\end{lstlisting}
Which is clearer when debugging or sharing code?  
Let the class debate and justify their preferences.
\end{teacherbox}

% ==============================================================
\end{document}

