\documentclass[12pt]{book}

% --- Packages ---
\usepackage[a4paper,margin=1in]{geometry}
\usepackage{graphicx}
\usepackage{hyperref}
\usepackage{fancyhdr}
\usepackage{titlesec}
\usepackage{listings}
\usepackage{xcolor}
\usepackage[utf8]{inputenc} % Ensure UTF-8 compatibility
\usepackage[T1]{fontenc}
\usepackage{url}
\usepackage{verbatim}
\sloppy % allows better line breaking

% --- Page style ---
\pagestyle{fancy}
\fancyhead[L]{ZyBooks Chapter 12 Companion}
\fancyhead[R]{\leftmark}
\fancyfoot[C]{\thepage}
\setlength{\headheight}{15pt}

% --- Listings configuration ---
\lstset{
  basicstyle=\ttfamily\footnotesize,
  backgroundcolor=\color{gray!10},
  frame=single,
  breaklines=true,
  showstringspaces=false,
  numbers=none,                % removed line numbers for easy copy/paste
  keywordstyle=\color{blue!70!black},
  commentstyle=\color{green!40!black},
  stringstyle=\color{orange!90!black},
  tabsize=4,
  keepspaces=true,
  columns=flexible,
  captionpos=b                 % captions below code blocks
}

% --- Metadata ---
\title{ZyBooks Chapter 12: Files in Python}
\author{Jeremy Evert}
\date{\today}

% ====================================================
\begin{document}

\frontmatter
\maketitle
\tableofcontents

\mainmatter

% chapters/chapter12.tex
\chapter{ZyBooks Chapter 12 Overview}
This chapter explores Python file operations, including reading, writing,
binary data, and file system interactions.

\chapter{12.1 Reading Files}

\section*{Overview}
In this section, students learn how to read from files using Python's built-in \texttt{open()} function. 
Files allow programs to save data permanently and later retrieve it. 
Instead of entering data manually each time a program runs, we can read information directly from a file.

\begin{quote}
\textbf{Learning goals:}
\begin{itemize}
  \item Understand how to open, read, and close text files in Python.
  \item Explore the difference between \texttt{read()}, \texttt{readline()}, and \texttt{readlines()}.
  \item Learn to process data from files (e.g., computing averages).
\end{itemize}
\end{quote}

---

\section{Reading from a File}

The most basic way to read data from a file is with \texttt{open()} and \texttt{read()}.

\begin{lstlisting}[language=Python, caption={Reading text from a file.}]
# Example 1: Reading the entire contents of a file

# Open the file in read mode
myjournal = open("journal.txt")

# Read the entire file into a single string
contents = myjournal.read()

# Display what was read
print(contents)

# Close the file after use
myjournal.close()
\end{lstlisting}

\noindent
\textbf{Key points:}
\begin{itemize}
  \item \texttt{open("filename")} creates a file object.
  \item \texttt{read()} reads all text at once and returns it as a string.
  \item Always close the file using \texttt{close()} when done.
\end{itemize}

---

\section{A More Complete Example}

This version adds print statements and clarifies program flow.

\begin{lstlisting}[language=Python, caption={Creating a file object and reading text.}]
print("Opening file myfile.txt.")
f = open("myfile.txt")  # create file object

print("Reading file myfile.txt.")
contents = f.read()     # read text into a string

print("Closing file myfile.txt.")
f.close()               # close the file

print("\nContents of myfile.txt:")
print(contents)
\end{lstlisting}

\textbf{Tip:} The file must be in the same directory as your Python script unless you specify a full path (e.g., \texttt{C:\textbackslash Users\textbackslash evertj\textbackslash myfile.txt}).

---

\section{Reading Line by Line}

The \texttt{readlines()} method reads each line into a list of strings.

\begin{lstlisting}[language=Python, caption={Reading all lines into a list.}]
# Example 2: Read lines from a file
my_file = open("readme.txt")
lines = my_file.readlines()

# Print the second line (remember, Python starts counting at 0)
print(lines[1])

my_file.close()
\end{lstlisting}

\textbf{Note:} Each element of \texttt{lines} includes the newline character \texttt{"\textbackslash n"}.

---

\section{Processing Data from a File}

Programs often read data from files to compute a result, such as an average.

\begin{lstlisting}[language=Python, caption={Calculating the average value of integers stored in a file.}]
# Example 3: Calculating an average from a file

print("Reading in data...")
f = open("mydata.txt")
lines = f.readlines()
f.close()

# Process data
print("\nCalculating average...")
total = 0
for ln in lines:
    total += int(ln)

avg = total / len(lines)
print(f"Average value: {avg}")
\end{lstlisting}

This example demonstrates:
\begin{itemize}
  \item How to iterate through file lines.
  \item Converting strings to integers using \texttt{int()}.
  \item Computing an average from numeric data.
\end{itemize}

---

\section{Iterating Directly Over a File Object}

Python lets you loop through a file directly, one line at a time.

\begin{lstlisting}[language=Python, caption={Iterating over the lines of a file.}]
"""Echo the contents of a file."""
f = open("myfile.txt")

for line in f:
    print(line, end="")  # end="" avoids double newlines

f.close()
\end{lstlisting}

This approach is memory-efficient and ideal for large files.

---

\section{Practice Exercise}

\textbf{Challenge:}  
Create a Python program that reads a filename from user input, opens that file, and prints its contents in uppercase.

\begin{lstlisting}[language=Python, caption={Challenge Activity: Read and modify file contents.}]
# Example 4: Read and transform file content
filename = input("Enter filename: ")

with open(filename) as f:       # 'with' auto-closes the file
    contents = f.read()

print(contents.upper())
\end{lstlisting}

---

\section{Explore More}

For additional reading and examples:
\begin{itemize}
  \item \href{https://docs.python.org/3/tutorial/inputoutput.html#reading-and-writing-files}{Python Documentation: Reading and Writing Files}
  \item \href{https://www.w3schools.com/python/python_file_handling.asp}{W3Schools: Python File Handling}
  \item \href{https://realpython.com/read-write-files-python/}{Real Python: Working with Files in Python}
\end{itemize}

---

\section*{Summary}
\begin{itemize}
  \item Use \texttt{open()} to access a file.
  \item \texttt{read()}, \texttt{readline()}, and \texttt{readlines()} offer flexibility.
  \item Always close files, or use the \texttt{with} statement.
  \item Practice reading, processing, and displaying file data.
\end{itemize}

\chapter{12.2 Writing Files}

\section*{Overview}

Reading files is like visiting the library---you take in information.  
Writing files, on the other hand, is like *becoming* the author.  
In this section, students learn how to create, modify, and save text files safely.  

Python’s built-in \texttt{open()} function provides multiple “modes” for writing, appending, and creating files.  
You’ll also learn why some write operations fail, how to avoid data loss, and how to make your programs polite authors who close their notebooks when finished.

\begin{quote}
\textbf{Learning goals:}
\begin{itemize}
  \item Understand how file modes (\texttt{w}, \texttt{a}, \texttt{x}, etc.) affect writing behavior.
  \item Learn to handle common write errors gracefully.
  \item Explore buffering and flushing to ensure data is saved properly.
  \item Appreciate the importance of file safety and reproducibility.
\end{itemize}
\end{quote}

---

\section{Basic File Writing}

The \texttt{write()} method records text into a file.  
Opening a file in mode \texttt{"w"} will create it if missing or overwrite it if it already exists.  
Think of it as starting a new diary page—sometimes that’s what you want, sometimes it’s heartbreak.

\begin{lstlisting}[language=Python, caption={Example 1: Writing text to a file.}]
def write_basic_example():
    """Write two lines to a new file."""
    with open("myfile.txt", "w", encoding="utf-8") as f:
        f.write("This is a brand-new file.\n")
        f.write("Second line: this one overwrites any previous content.\n")
    print("[Success] File written successfully!")

write_basic_example()
\end{lstlisting}

\begin{quote}
\textbf{Sample File Output:} \texttt{myfile.txt}
\begin{verbatim}
This is a brand-new file.
Second line: this one overwrites any previous content.
\end{verbatim}
\end{quote}

---

\section{Why Some Writes Fail}

You can only write strings to text files.  
Attempting to write a number directly will cause Python to raise a dramatic \texttt{TypeError}.

\begin{lstlisting}[language=Python, caption={Example 2: Handling write errors with style.}]
def demonstrate_wrong_write():
    """Show why writing numbers directly causes a TypeError."""
    try:
        with open("wrong_write.txt", "w", encoding="utf-8") as f:
            # This will fail: write() only accepts strings.
            f.write(3.14159)
    except TypeError as e:
        print("[Error]", e)
        print("[Hint] Convert numbers to strings using str() or f-strings.")

    # Correct version
    with open("right_write.txt", "w", encoding="utf-8") as f:
        num1, num2 = 5, 7.5
        f.write(f"{num1} + {num2} = {num1 + num2}\n")
    print("[Fixed] Math written successfully!")

demonstrate_wrong_write()
\end{lstlisting}

---

\section{File Modes}

File modes are like the different moods of a writer—each one changes the tone of what happens next.

\begin{center}
\begin{tabular}{l l l l l}
\textbf{Mode} & \textbf{Description} & \textbf{Read?} & \textbf{Write?} & \textbf{Overwrite?} \\
\hline
\texttt{r} & Read only & Yes & No & No \\
\texttt{w} & Write (overwrite) & No & Yes & Yes \\
\texttt{a} & Append to end & No & Yes & No \\
\texttt{r+} & Read and write (must exist) & Yes & Yes & No \\
\texttt{w+} & Read and write (truncates) & Yes & Yes & Yes \\
\texttt{a+} & Read and append & Yes & Yes & No \\
\texttt{x} & Create new file (error if exists) & No & Yes & N/A \\
\end{tabular}
\end{center}

\begin{lstlisting}[language=Python, caption={Example 3: Trying wrong modes, then fixing them.}]
from pathlib import Path
import io

def show_file(path):
    p = Path(path)
    print(f"\n[{p.name}] contents:")
    print(p.read_text(encoding="utf-8") if p.exists() else "<missing>")

def demonstrate_modes():
    path = "modes_demo.txt"
    Path(path).write_text("START\n", encoding="utf-8")

    # Wrong: open in read mode, try to write
    try:
        with open(path, "r", encoding="utf-8") as f:
            f.write("APPEND\n")
    except io.UnsupportedOperation as e:
        print("[Error]", e)
        print("[Hint] 'r' is read-only; use 'a' for append.")

    # Correct: append mode
    with open(path, "a", encoding="utf-8") as f:
        f.write("APPEND\n")

    show_file(path)

demonstrate_modes()
\end{lstlisting}

---

\section{Append Mode: The Diary Approach}

The \texttt{a} mode appends to an existing file—great for logs, journals, and confessions you don’t want erased.

\begin{lstlisting}[language=Python, caption={Example 4: Appending to a file repeatedly.}]
from datetime import datetime

def append_to_log(entry, filename="daily_log.txt"):
    """Append timestamped entries to a log file."""
    with open(filename, "a", encoding="utf-8") as f:
        timestamp = datetime.now().strftime("%Y-%m-%d %H:%M:%S")
        f.write(f"[{timestamp}] {entry}\n")
    print("[Info] Log entry added.")

# Let's test a few entries
append_to_log("Started Chapter 12 examples.")
append_to_log("Tried append mode. It worked!")
append_to_log("Feeling confident about file handling.")
\end{lstlisting}

---

\section{Buffered Output and Flushing}

When you write to a file, Python first stores the data in memory before writing it to disk.  
This is called *buffering*---it makes writing faster, but also means data might not appear immediately.

\begin{lstlisting}[language=Python, caption={Example 5: Forcing a buffer flush.}]
import os, time
from pathlib import Path

def buffering_demo(path="buffer_demo.txt"):
    p = Path(path)
    if p.exists():
        p.unlink()

    f = open(path, "w", encoding="utf-8")
    f.write("Write me (still in memory)...")
    print("[Info] File opened and data written (but not yet saved).")
    time.sleep(1)
    f.flush()          # Push data from Python to OS
    os.fsync(f.fileno())  # Ensure the OS writes it to disk
    print("[Success] Data flushed to disk.")
    f.close()

buffering_demo()
\end{lstlisting}

---

\section{Safe File Creation}

The \texttt{"x"} mode is the “no overwrite allowed” option.  
It’s perfect for protecting students’ lab reports, thesis drafts, or personal manifestos.

\begin{lstlisting}[language=Python, caption={Example 6: Using 'x' mode to prevent overwrite.}]
def create_once(filename="create_once.txt"):
    try:
        with open(filename, "x", encoding="utf-8") as f:
            f.write("This file will never be overwritten.\n")
        print(f"[Created] New file: {filename}")
    except FileExistsError:
        print(f"[Skipped] '{filename}' already exists; not overwritten.")

create_once()
create_once()
\end{lstlisting}

---

\section{Fun Activity: The Compliment Machine}

Let’s make something a bit sillier---a program that takes user input and writes  
each compliment to a file so you can re-read your greatness later.

\begin{lstlisting}[language=Python, caption={Example 7: The Compliment Machine.}]
def compliment_machine():
    filename = "compliments.txt"
    print("Welcome to the Compliment Machine!")
    print("Type compliments to save them; press Enter on an empty line to quit.\n")

    with open(filename, "a", encoding="utf-8") as f:
        while True:
            compliment = input("Say something nice: ")
            if not compliment.strip():
                break
            f.write(compliment + "\n")
            print("[Saved] Compliment added!\n")

    print(f"All compliments saved to {filename}.")

compliment_machine()
\end{lstlisting}

---

\section*{Summary}
\begin{itemize}
  \item Mode \texttt{"w"} overwrites, \texttt{"a"} appends, and \texttt{"x"} creates new.
  \item \texttt{write()} only accepts strings; use \texttt{str()} or f-strings for other data.
  \item Buffered writes may not appear immediately—use \texttt{flush()} or close the file.
  \item The safest way to write: \texttt{with open(...)} ensures automatic cleanup.
  \item Writing to files is how programs tell stories. Make yours a good one.
\end{itemize}

\chapter{12.3 Interacting With File Systems}

\section*{Big Picture Mental Model}

When your program touches the file system, several layers cooperate:

\begin{enumerate}
  \item \textbf{Your Python code} calls functions like \texttt{open()}, \texttt{os.stat()}, \texttt{os.remove()}, \texttt{pathlib.Path(...)}.
  \item \textbf{CPython implementation} translates those calls into C functions that use the operating system's native API.
        On Linux/macOS this is usually POSIX calls (open, read, write, stat, unlink).
        On Windows it goes through the Win32 layer (CreateFileW, ReadFile, WriteFile, GetFileInformationByHandle, DeleteFileW).
  \item \textbf{The operating system kernel} checks permissions, updates metadata, and interacts with the file system driver.
        It also uses caches and scheduling to read/write blocks on storage devices.
  \item \textbf{Hardware and firmware} (disk controller, SSD firmware, DMA) actually move bytes.
        \textbf{CPUs (Intel/AMD/ARM)} execute instructions, switch between user mode and kernel mode on system calls,
        and provide memory management and caching. The CPU does not know about "files" directly; it executes the OS code that does.
\end{enumerate}

\noindent
Takeaway: Python gives a friendly interface, but durability, permissions, and path rules come from the OS and file system.

\begin{quote}
\textbf{Docs to Bookmark}
\begin{itemize}
  \item Python Tutorial: Reading and Writing Files -- \href{https://docs.python.org/3/tutorial/inputoutput.html#reading-and-writing-files}{docs.python.org}
  \item \texttt{os} module -- \href{https://docs.python.org/3/library/os.html}{docs.python.org}
  \item \texttt{os.path} module -- \href{https://docs.python.org/3/library/os.path.html}{docs.python.org}
  \item \texttt{pathlib} -- \href{https://docs.python.org/3/library/pathlib.html}{docs.python.org}
  \item \texttt{shutil} (copy, move) -- \href{https://docs.python.org/3/library/shutil.html}{docs.python.org}
  \item File object methods (\texttt{flush}) -- \href{https://docs.python.org/3/library/io.html}{docs.python.org}
\end{itemize}
\end{quote}

\section{Portable Paths: os.path.join and pathlib}

Hard-coding backslashes (Windows) or slashes (Linux/macOS) makes code fragile. Use joiners.

\begin{lstlisting}[language=Python, caption={Right vs. wrong for building file paths.}]
import os
from pathlib import Path

def build_paths():
    # WRONG on non-Windows and brittle even on Windows:
    p_bad = "logs\\2025\\01\\log.txt"   # backslashes are Windows-only
    print("Brittle:", p_bad)

    # RIGHT: OS-appropriate separator via os.path.join
    p_good = os.path.join("logs", "2025", "01", "log.txt")
    print("Portable:", p_good)

    # RIGHT: Path objects are even nicer
    p = Path("logs") / "2025" / "01" / "log.txt"
    print("Pathlib:", str(p))

build_paths()
\end{lstlisting}

\noindent
Windows note: inside Python string literals, a single backslash begins an escape (like \texttt{"\textbackslash n"}). Use raw strings like \texttt{r"C:\textbackslash users\textbackslash me"} or double the backslashes.

\section{Existence, File vs. Directory, and Getting Size}

\begin{lstlisting}[language=Python, caption={Check existence and type with both os.path and pathlib.}]
import os
from pathlib import Path

def existence_and_type(path_str: str):
    print("\n-- Using os.path --")
    print("exists:", os.path.exists(path_str))
    print("isfile:", os.path.isfile(path_str))
    print("isdir:", os.path.isdir(path_str))

    print("\n-- Using pathlib --")
    p = Path(path_str)
    print("exists:", p.exists())
    print("is_file:", p.is_file())
    print("is_dir:", p.is_dir())

    if p.exists():
        print("size:", p.stat().st_size, "bytes")

existence_and_type("modes_demo.txt")
existence_and_type("logs")
\end{lstlisting}

\section{Metadata: os.stat and datetime}

\begin{lstlisting}[language=Python, caption={Inspect file metadata and pretty-print timestamps.}]
import os, datetime
from pathlib import Path

def show_stat(path_str: str):
    p = Path(path_str)
    if not p.exists():
        print(f"{path_str!r} does not exist")
        return
    st = p.stat()  # same as os.stat(path_str)
    print("\n-- stat for", path_str, "--")
    print("size:", st.st_size, "bytes")
    print("mode (permission bits):", oct(st.st_mode))
    print("modified:", datetime.datetime.fromtimestamp(st.st_mtime))
    print("created (platform dependent):", datetime.datetime.fromtimestamp(st.st_ctime))

show_stat("modes_demo.txt")
\end{lstlisting}

\noindent
Platform note: \texttt{st\_ctime} is creation time on Windows, but on POSIX it is "metadata change" time.

\section{Walking a Directory Tree}

\begin{lstlisting}[language=Python, caption={Walk with os.walk and filter by extension.}]
import os
from pathlib import Path

def list_py_files(root=".", ext=".txt"):
    print(f"\nListing {ext} files under {root!r}")
    for dirpath, subdirs, files in os.walk(root):
        for name in files:
            if name.lower().endswith(ext):
                print(os.path.join(dirpath, name))

list_py_files("logs", ".txt")
\end{lstlisting}

\noindent
\texttt{os.walk} yields a 3-tuple per directory. The heavy lifting (reading directory entries) is done by the OS; Python iterates and filters.

\section{Creating, Renaming, Copying, Deleting}

\begin{lstlisting}[language=Python, caption={Safe create, rename, copy, and delete with error handling.}]
import shutil
from pathlib import Path

def safe_create_dir(path: str):
    Path(path).mkdir(parents=True, exist_ok=True)
    print("Ensured directory exists:", path)

def safe_rename(src: str, dst: str):
    try:
        # os.replace is atomic when src and dst are on the same filesystem
        os.replace(src, dst)
        print(f"Renamed {src!r} -> {dst!r}")
    except FileNotFoundError:
        print("Cannot rename: source not found.")
    except PermissionError:
        print("Cannot rename: permission denied.")

def safe_copy(src: str, dst: str):
    try:
        shutil.copy2(src, dst)  # preserves timestamps and metadata where possible
        print(f"Copied {src!r} -> {dst!r}")
    except FileNotFoundError:
        print("Cannot copy: source not found.")
    except PermissionError:
        print("Cannot copy: permission denied.")

def safe_delete(path: str):
    try:
        Path(path).unlink()
        print("Deleted file:", path)
    except FileNotFoundError:
        print("Nothing to delete:", path)
    except IsADirectoryError:
        print("Path is a directory; use rmdir or shutil.rmtree.")
    except PermissionError:
        print("Cannot delete: permission denied.")

safe_create_dir("sandbox")
Path("sandbox/demo.txt").write_text("hello\n", encoding="utf-8")
safe_copy("sandbox/demo.txt", "sandbox/demo_copy.txt")
safe_rename("sandbox/demo_copy.txt", "sandbox/demo_moved.txt")
safe_delete("sandbox/demo_moved.txt")
\end{lstlisting}

\noindent
Atomicity note: \texttt{os.replace} is designed to be atomic on the same filesystem volume. If you move across drives, use \texttt{shutil.move} which may copy then delete.

\section{Portable File Path Building Activity}

\begin{lstlisting}[language=Python, caption={Demonstrate os.path.join results on different OSes.}]
import os

def join_examples():
    a = os.path.join("subdir", "output.txt")
    b = os.path.join("sounds", "cars", "honk.mp3")
    print("Example join A:", a)
    print("Example join B:", b)
    print("Path separator on this OS:", os.path.sep)

join_examples()
\end{lstlisting}

\section{Splitting Paths and Getting Extensions}

\begin{lstlisting}[language=Python, caption={Split with os.path.split and get extension with splitext.}]
import os

def split_examples(p: str):
    head, tail = os.path.split(p)
    root, ext = os.path.splitext(p)
    print("\nSplit:", p)
    print(" head:", head)
    print(" tail:", tail)
    print(" root:", root)
    print(" ext:", ext)

split_examples(os.path.join("C:\\", "Users", "Demo", "batsuit.jpg"))
\end{lstlisting}

\section{Challenge: Use os.walk to Count Specific Files}

\begin{lstlisting}[language=Python, caption={Count .txt files and handle permissions gracefully.}]
import os

def count_ext(root: str, ext: str = ".txt") -> int:
    total = 0
    for dirpath, subdirs, files in os.walk(root, onerror=None):
        for name in files:
            if name.lower().endswith(ext):
                total += 1
    return total

print("Number of .txt files under logs:", count_ext("logs", ".txt"))
\end{lstlisting}

\section{Safe and Durable Writes: Temp File + Atomic Replace}

\begin{lstlisting}[language=Python, caption={Avoid partial writes by writing to a temp file and replacing.}]
import os, tempfile
from pathlib import Path

def atomic_write_text(path: str, text: str):
    target = Path(path)
    target.parent.mkdir(parents=True, exist_ok=True)

    # Create a temp file in the same directory to keep the replace atomic
    with tempfile.NamedTemporaryFile("w", encoding="utf-8", dir=str(target.parent), delete=False) as tmp:
        tmp.write(text)
        tmp.flush()
        os.fsync(tmp.fileno())  # push to disk as best as the OS can

        tmp_name = tmp.name

    # Replace is atomic on same filesystem; readers will see old or new, not partial
    os.replace(tmp_name, str(target))
    print("Atomically wrote:", target)

atomic_write_text("sandbox/report.txt", "final contents\n")
\end{lstlisting}

\noindent
Durability note: \texttt{flush()} moves data from Python to the OS; \texttt{os.fsync()} asks the OS to persist to storage. On real hardware, disk caches and controllers also play a role. If the computer loses power, even fsync cannot guarantee survival on every device, but it is the standard tool for best-effort durability.

\section{Windows-Specific Notes}

\begin{itemize}
  \item Path length: old Windows APIs had a 260-character limit. Modern Windows can support longer paths with configuration; the prefix \texttt{\\\\?\\} can be involved under the hood. Pathlib and modern Python try to handle this for you.
  \item Drives and UNC: paths can be drive-based (C:\textbackslash) or UNC (\texttt{\\\\server\\share\\path}). \texttt{pathlib.Path} handles both.
  \item Newlines: text mode translates newlines to the OS convention. Use binary mode (\texttt{"rb"/"wb"}) if you need raw bytes.
  \item Case: Windows file systems are usually case-insensitive but case-preserving; Linux is case-sensitive.
\end{itemize}

\section{Right vs. Wrong: OS Operations With Explanations}

\begin{lstlisting}[language=Python, caption={Demonstrate typical mistakes and show the fixes.}]
import os, io
from pathlib import Path

def show(path):
    p = Path(path)
    print(f"[{path}] exists:", p.exists(), "is_file:", p.is_file(), "is_dir:", p.is_dir())

def wrong_then_right_remove():
    # Wrong: try to unlink a directory with unlink
    safe_create_dir("sandbox_dir")
    try:
        Path("sandbox_dir").unlink()
    except IsADirectoryError as e:
        print("Caught:", e)
        print("Reason: unlink removes files, not directories.")
    # Right:
    try:
        os.rmdir("sandbox_dir")
        print("Removed empty directory 'sandbox_dir'")
    except OSError as e:
        print("Directory not empty; use shutil.rmtree if needed.")

def wrong_then_right_open_dir():
    # Wrong: open() expects files, not directories
    safe_create_dir("sandbox_dir2")
    try:
        open("sandbox_dir2", "r")
    except IsADirectoryError as e:
        print("Caught:", e)
        print("Reason: open() cannot open directories in text mode.")
    # Right: list directory entries
    print("Entries:", list(os.scandir("sandbox_dir2")))
    os.rmdir("sandbox_dir2")

wrong_then_right_remove()
wrong_then_right_open_dir()
\end{lstlisting}

\section{Pathlib Cheatsheet}

\begin{lstlisting}[language=Python, caption={Common pathlib operations, very readable.}]
from pathlib import Path

p = Path("logs") / "2025" / "01" / "log.txt"
print("Parent:", p.parent)            # logs/2025/01
print("Name:", p.name)                # log.txt
print("Stem:", p.stem)                # log
print("Suffix:", p.suffix)            # .txt

p.parent.mkdir(parents=True, exist_ok=True)
p.write_text("hello\n", encoding="utf-8")
print("Read back:", p.read_text(encoding="utf-8"))

for q in p.parent.rglob("*.txt"):
    print("Found:", q)
\end{lstlisting}

\section*{Why This Works The Way It Works}

\begin{itemize}
  \item \textbf{System calls}: File operations cross from user mode to kernel mode through system calls. The CPU handles this transition (for x86, via syscall/sysenter or legacy int 0x80), then resumes your code when the OS returns. The CPU is agnostic about files; it only runs instructions.
  \item \textbf{Caching layers}: The OS keeps a page cache to avoid slow disk I/O. That is why \texttt{write()} may not be visible until newline, flush, close, or after a delay. \texttt{os.fsync()} asks the OS to flush its cache to the storage driver.
  \item \textbf{File systems}: NTFS, APFS, ext4, and others decide naming rules, metadata, and durability guarantees. Python does not change these rules; it exposes them.
  \item \textbf{Portability}: Using \texttt{os.path.join} or \texttt{pathlib} and handling exceptions (\texttt{FileNotFoundError}, \texttt{PermissionError}, \texttt{IsADirectoryError}, \texttt{NotADirectoryError}) produces code that behaves well across OSes.
\end{itemize}

\section*{Practice Prompts For Students}

\begin{enumerate}
  \item Build a portable path for today's date, such as:
  \verb|logs/YYYY/MM/DD/log.txt|. \\
  Create any missing directories and write one line safely to that file.

  \item Walk a directory tree and print the three largest files by size. \\
  Explain how you computed file sizes using the \verb|os.stat()| function.

  \item Write a function \verb|safe_replace(path, text)| that writes to a temporary file
  and atomically replaces the target file, then verify that the contents were updated correctly.

  \item On Windows, demonstrate the difference between a *raw string*
  (e.g., \verb|r"C:\new\logs"|) and an *escaped string*
  (e.g., \verb|"C:\\new\\logs"|). \\
  Explain what happens with backslashes in each case.
\end{enumerate}


\chapter{Binary Data}

\section*{Binary Data Basics}
Some files consist of data stored as a sequence of bytes, known as \textbf{binary data}, 
that is not encoded into readable text using encodings like ASCII or UTF-8. 
Examples include images, videos, and PDFs.

When opened in a text editor, binary files often appear as random or unreadable symbols because 
the editor is trying to interpret raw byte values as text characters.

\texttt{bytes} objects are used in Python to represent sequences of byte values. 
They are immutable (cannot be changed after creation), similar to strings. 
A bytes object can be created using the built-in \texttt{bytes()} function or a bytes literal.

\begin{itemize}
  \item \texttt{bytes("A text string", "ascii")} – creates bytes from a string using ASCII encoding
  \item \texttt{bytes(100)} – creates 100 zero-value bytes
  \item \texttt{bytes([12, 15, 20])} – creates bytes from numeric values
\end{itemize}

You can also create a bytes literal by prefixing a string with \texttt{b}:

\begin{lstlisting}[language=Python, caption={Creating a bytes object using a literal.}]
my_bytes = b"This is a bytes literal"
print(my_bytes)
print(type(my_bytes))
\end{lstlisting}

\begin{verbatim}
b'This is a bytes literal'
<class 'bytes'>
\end{verbatim}

\section*{Byte String Literals}
You can represent specific byte values using hexadecimal escape codes. 
Each \texttt{\textbackslash xHH} represents one byte in hexadecimal form.

\begin{lstlisting}[language=Python, caption={Byte string literals.}]
print(b"123456789" == b"\x31\x32\x33\x34\x35\x36\x37\x38\x39")
\end{lstlisting}

\begin{verbatim}
True
\end{verbatim}

\section*{Reading and Writing Binary Files}
When working with binary files, use \texttt{'rb'} (read binary) or \texttt{'wb'} (write binary) modes.

\begin{lstlisting}[language=Python, caption={Opening binary files.}]
# Open file for binary reading
f = open("data.bin", "rb")
contents = f.read()
f.close()

# Open file for binary writing
f = open("new_data.bin", "wb")
f.write(b"\x01\x02\x03\x04")
f.close()
\end{lstlisting}

In binary mode, Python does not translate newline characters. 
On Windows, this avoids converting \texttt{\textbackslash n} to \texttt{\textbackslash r\textbackslash n}.

\section*{Inspecting Binary Contents of a File}
Suppose we have an image \texttt{ball.bmp}. 
Reading it in binary mode allows us to inspect the raw byte values.

\begin{lstlisting}[language=Python, caption={Inspecting binary contents of an image.}]
f = open("ball.bmp", "rb")
contents = f.read(32)
f.close()

print("First 32 bytes of ball.bmp:")
print(contents)
\end{lstlisting}

This prints unreadable byte sequences like:
\begin{verbatim}
b'BM\xf6\x00\x00\x00\x00\x00\x00\x00\x06\x04\x00\x00...'
\end{verbatim}

\section*{Example: Altering a BMP Image}
\begin{lstlisting}[language=Python, caption={Modifying pixels in a BMP image.}]
import struct

ball_file = open("ball.bmp", "rb")
ball_data = ball_file.read()
ball_file.close()

# BMP header stores pixel data offset in bytes 10–14
pixel_data_loc = struct.unpack("<I", ball_data[10-14])[0]

# Replace 3000 pixels with red, green, yellow pattern
new_pixels = b"\x01" * 3000 + b"\x02" * 3000 + b"\x03" * 3000

# Create new image data
new_data = ball_data[:pixel_data_loc] + new_pixels + ball_data[pixel_data_loc + len(new_pixels):]

# Save altered image
with open("new_ball.bmp", "wb") as f:
    f.write(new_data)
\end{lstlisting}

\section*{Using the struct Module}
The \texttt{struct} module helps pack and unpack data into byte sequences.

\begin{lstlisting}[language=Python, caption={Packing and unpacking bytes.}]
import struct

# Pack integers into binary data
data = struct.pack(">hh", 5, 256)
print("Packed:", data)

# Unpack binary back to integers
unpacked = struct.unpack(">hh", data)
print("Unpacked:", unpacked)
\end{lstlisting}

\begin{verbatim}
Packed: b'\x00\x05\x01\x00'
Unpacked: (5, 256)
\end{verbatim}

\section*{Performance Comparison: Binary vs Text Write Speed}
Let’s see how fast binary writing can be compared to text writing.

\begin{lstlisting}[language=Python, caption={Comparing binary and text file speeds.}]
import time
import os

data_size = 10_000_000  # 10 MB
text_data = "A" * data_size
binary_data = b"A" * data_size

# Write text data
start = time.time()
with open("text_test.txt", "w") as f:
    f.write(text_data)
text_time = time.time() - start

# Write binary data
start = time.time()
with open("binary_test.bin", "wb") as f:
    f.write(binary_data)
binary_time = time.time() - start

print(f"Text write: {text_time:.4f} s")
print(f"Binary write: {binary_time:.4f} s")
print(f"Binary is {(text_time/binary_time):.2f}x faster!")
\end{lstlisting}

Depending on your storage and system, binary writes are often slightly faster 
because fewer conversions are performed during I/O operations.

\section*{Key Takeaways}
\begin{itemize}
  \item Binary files store raw bytes, not human-readable text.
  \item Use \texttt{rb} / \texttt{wb} modes for reading and writing binary files.
  \item The \texttt{struct} module helps encode/decode structured binary data.
  \item Binary I/O can be faster than text I/O for large datasets.
\end{itemize}

\chapter{12.5 Command-Line Arguments and Files}

\section*{Overview}
Command-line arguments let your program take input file names or other parameters
directly from the shell.  
It’s how you tell your code: *“Work with this file today, not the one you hard-coded last night.”*

\begin{quote}
\textbf{Docs to Bookmark:}
\begin{itemize}
  \item \href{https://docs.python.org/3/library/sys.html}{sys module}
  \item \href{https://docs.python.org/3/library/os.path.html}{os.path module}
\end{itemize}
\end{quote}

---

\section{Example: Using an Input File from the Command Line}

\begin{lstlisting}[language=Python, caption={listing\_5\_1\_command\_line\_args.py}]
import sys
import os

if len(sys.argv) != 2:
    print(f"Usage: python {sys.argv[0]} input_file")
    sys.exit(1)  # 1 = error

file_name = sys.argv[1]
print(f"Opening file {file_name}...")

if not os.path.exists(file_name):
    print("File does not exist.")
    sys.exit(1)

with open(file_name, "r", encoding="utf-8") as f:
    print("Reading two integers.")
    num1 = int(f.readline())
    num2 = int(f.readline())

print(f"Closing file {file_name}.")
print(f"\nnum1: {num1}")
print(f"num2: {num2}")
print(f"num1 + num2 = {num1 + num2}")
\end{lstlisting}

\noindent
\textbf{Run it like this:}

\begin{verbatim}
> python listing_5_1_command_line_args.py myfile1.txt
> python listing_5_1_command_line_args.py myfile2.txt
> python listing_5_1_command_line_args.py missing.txt
\end{verbatim}

\noindent
Output examples:
\begin{verbatim}
Opening file myfile1.txt...
Reading two integers.
Closing file myfile1.txt.

num1: 5
num2: 10
num1 + num2 = 15
\end{verbatim}

---

\section{Files for Testing}

\textbf{myfile1.txt}
\begin{verbatim}
5
10
\end{verbatim}

\textbf{myfile2.txt}
\begin{verbatim}
-34
7
\end{verbatim}

---

\section{Extending the Idea}
\begin{itemize}
  \item Add a second argument for an \emph{output file}.
  \item Add try/except to handle non-numeric data.
  \item Use \texttt{argparse} for fancier options later in the course.
\end{itemize}

\section*{Practice Prompts}
\begin{enumerate}
  \item Modify the program so it accepts both an input and output filename:
  \verb|python myscript.py infile.txt outfile.txt|
  \item For a run as \verb|python scriptname data.txt|, what is \verb|sys.argv[1]|?
  \item What happens if you forget the argument? Why does Python show a usage message?
\end{enumerate}

\chapter{12.6 The \texttt{with} Statement}

\section*{Overview}
A \texttt{with} statement is Python’s built-in way of managing resources — such as files — safely and elegantly.  
When used with \texttt{open()}, it guarantees that the file is closed automatically when the block of code completes,  
even if an exception occurs inside the block.

Think of it as a friendly librarian: you borrow a book (open a file), do your reading,  
and no matter what happens — whether you finish or drop your coffee — the librarian takes the book back and shelves it neatly.

\begin{quote}
\textbf{Docs to Bookmark:}
\begin{itemize}
  \item \href{https://docs.python.org/3/reference/compound_stmts.html#the-with-statement}{The with statement (Python Docs)}
  \item \href{https://docs.python.org/3/tutorial/inputoutput.html#methods-of-file-objects}{File objects in Python}
\end{itemize}
\end{quote}

---

\section{Basic Example: Safe File Reading}

\begin{lstlisting}[language=Python, caption={Opening and reading a file safely.}]
print("Opening myfile.txt")
with open("myfile.txt", "r", encoding="utf-8") as f:
    contents = f.read()
    print(contents)
print("File closed automatically after this block!")
\end{lstlisting}

\noindent
The \texttt{with} statement ensures that once you exit the block, \texttt{f.close()} is called automatically.  
Even if an error occurs inside the block, Python still closes the file properly.

---

\section{Why Not Just Call \texttt{close()} Yourself?}

\begin{lstlisting}[language=Python, caption={Manual open and close -- easy to forget.}]
f = open("oops.txt", "w", encoding="utf-8")
f.write("This file might stay open forever...")
# Oops! We forgot f.close()
\end{lstlisting}

\noindent
If you forget to close a file, it can remain locked or unflushed in memory —  
causing confusion, corrupted files, or grumpy system administrators.  

\begin{lstlisting}[language=Python, caption={Fixed using a with statement.}]
with open("oops.txt", "w", encoding="utf-8") as f:
    f.write("This version closes itself. Much better!")
\end{lstlisting}

---

\section{Reading and Writing Together}

\begin{lstlisting}[language=Python, caption={Using the same file object for reading and writing.}]
print("Opening numbers.txt for reading and writing...")
with open("numbers.txt", "r+", encoding="utf-8") as f:
    num1 = int(f.readline())
    num2 = int(f.readline())
    product = num1 * num2
    f.write(f"\nProduct: {product}\n")
print("Closed numbers.txt automatically.")
\end{lstlisting}

---

\section{Cautionary Tale: The File That Refused to Close}

\begin{lstlisting}[language=Python, caption={Why with is safer.}]
try:
    f = open("chaos.txt", "w", encoding="utf-8")
    f.write("Everything is fine so far...")
    raise RuntimeError("🔥 Oops! Chaos strikes!")
    f.close()
except Exception as e:
    print("Caught error:", e)
    print("Did we close the file? Nope!")

# Now the safe way:
try:
    with open("calm.txt", "w", encoding="utf-8") as f:
        f.write("Tranquility restored. File will close no matter what.")
        raise RuntimeError("🌪 A wild exception appears!")
except Exception as e:
    print("Caught safely:", e)
\end{lstlisting}

\noindent
Even though both blocks raise errors, only the second one guarantees that the file is properly closed.

---

\section{Challenge Example: Writing Logs}

\begin{lstlisting}[language=Python, caption={Appending a timestamped log entry safely.}]
from datetime import datetime

def log_event(message):
    """Append an event to a log file with timestamp."""
    timestamp = datetime.now().strftime("%Y-%m-%d %H:%M:%S")
    with open("log.txt", "a", encoding="utf-8") as log:
        log.write(f"[{timestamp}] {message}\n")

log_event("Program started successfully.")
log_event("User pressed the red button.")
\end{lstlisting}

---

\section{Quiz Questions}
\begin{enumerate}
  \item When does Python automatically close a file opened with \texttt{with}?  
        \textbf{Answer:} When the indented block ends, even if an error occurs.
  \item What is a \texttt{context manager}?  
        \textbf{Answer:} An object that sets up and tears down resources automatically.
  \item What happens if you forget \texttt{close()}?  
        \textbf{Answer:} The file may remain locked or data may not be saved.
\end{enumerate}

---

\section*{Summary}
\begin{itemize}
  \item Use \texttt{with open(...)} to ensure automatic closure of files.
  \item It’s cleaner, safer, and preferred in all modern Python code.
  \item Works great for other resources too (network sockets, database connections, etc.).
\end{itemize}

\begin{quote}
\textbf{Moral of the story:}  
If your program opens files without \texttt{with}, it’s like leaving the refrigerator door open — it still “works,” but you’re wasting power and asking for trouble.
\end{quote}



\backmatter
\chapter*{Notes}
This companion book was designed to accompany the zyBooks interactive textbook, Chapter~12: \textit{Files in Python}.  
Each section includes clear examples that are ready to copy and paste directly from this PDF into your favorite code editor.  
The examples also include sample data files, demonstrations of modern best practices, and a touch of humor to keep learning lively.

\end{document}
