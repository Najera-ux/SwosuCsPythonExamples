\chapter{12.1 Reading Files}

\section*{Overview}
In this section, students learn how to read from files using Python's built-in \texttt{open()} function. 
Files allow programs to save data permanently and later retrieve it. 
Instead of entering data manually each time a program runs, we can read information directly from a file.

\begin{quote}
\textbf{Learning goals:}
\begin{itemize}
  \item Understand how to open, read, and close text files in Python.
  \item Explore the difference between \texttt{read()}, \texttt{readline()}, and \texttt{readlines()}.
  \item Learn to process data from files (e.g., computing averages).
\end{itemize}
\end{quote}

---

\section{Reading from a File}

The most basic way to read data from a file is with \texttt{open()} and \texttt{read()}.

\begin{lstlisting}[language=Python, caption={Reading text from a file.}]
# Example 1: Reading the entire contents of a file

# Open the file in read mode
myjournal = open("journal.txt")

# Read the entire file into a single string
contents = myjournal.read()

# Display what was read
print(contents)

# Close the file after use
myjournal.close()
\end{lstlisting}

\noindent
\textbf{Key points:}
\begin{itemize}
  \item \texttt{open("filename")} creates a file object.
  \item \texttt{read()} reads all text at once and returns it as a string.
  \item Always close the file using \texttt{close()} when done.
\end{itemize}

---

\section{A More Complete Example}

This version adds print statements and clarifies program flow.

\begin{lstlisting}[language=Python, caption={Creating a file object and reading text.}]
print("Opening file myfile.txt.")
f = open("myfile.txt")  # create file object

print("Reading file myfile.txt.")
contents = f.read()     # read text into a string

print("Closing file myfile.txt.")
f.close()               # close the file

print("\nContents of myfile.txt:")
print(contents)
\end{lstlisting}

\textbf{Tip:} The file must be in the same directory as your Python script unless you specify a full path (e.g., \texttt{C:\textbackslash Users\textbackslash evertj\textbackslash myfile.txt}).

---

\section{Reading Line by Line}

The \texttt{readlines()} method reads each line into a list of strings.

\begin{lstlisting}[language=Python, caption={Reading all lines into a list.}]
# Example 2: Read lines from a file
my_file = open("readme.txt")
lines = my_file.readlines()

# Print the second line (remember, Python starts counting at 0)
print(lines[1])

my_file.close()
\end{lstlisting}

\textbf{Note:} Each element of \texttt{lines} includes the newline character \texttt{"\textbackslash n"}.

---

\section{Processing Data from a File}

Programs often read data from files to compute a result, such as an average.

\begin{lstlisting}[language=Python, caption={Calculating the average value of integers stored in a file.}]
# Example 3: Calculating an average from a file

print("Reading in data...")
f = open("mydata.txt")
lines = f.readlines()
f.close()

# Process data
print("\nCalculating average...")
total = 0
for ln in lines:
    total += int(ln)

avg = total / len(lines)
print(f"Average value: {avg}")
\end{lstlisting}

This example demonstrates:
\begin{itemize}
  \item How to iterate through file lines.
  \item Converting strings to integers using \texttt{int()}.
  \item Computing an average from numeric data.
\end{itemize}

---

\section{Iterating Directly Over a File Object}

Python lets you loop through a file directly, one line at a time.

\begin{lstlisting}[language=Python, caption={Iterating over the lines of a file.}]
"""Echo the contents of a file."""
f = open("myfile.txt")

for line in f:
    print(line, end="")  # end="" avoids double newlines

f.close()
\end{lstlisting}

This approach is memory-efficient and ideal for large files.

---

\section{Practice Exercise}

\textbf{Challenge:}  
Create a Python program that reads a filename from user input, opens that file, and prints its contents in uppercase.

\begin{lstlisting}[language=Python, caption={Challenge Activity: Read and modify file contents.}]
# Example 4: Read and transform file content
filename = input("Enter filename: ")

with open(filename) as f:       # 'with' auto-closes the file
    contents = f.read()

print(contents.upper())
\end{lstlisting}

---

\section{Explore More}

For additional reading and examples:
\begin{itemize}
  \item \href{https://docs.python.org/3/tutorial/inputoutput.html#reading-and-writing-files}{Python Documentation: Reading and Writing Files}
  \item \href{https://www.w3schools.com/python/python_file_handling.asp}{W3Schools: Python File Handling}
  \item \href{https://realpython.com/read-write-files-python/}{Real Python: Working with Files in Python}
\end{itemize}

---

\section*{Summary}
\begin{itemize}
  \item Use \texttt{open()} to access a file.
  \item \texttt{read()}, \texttt{readline()}, and \texttt{readlines()} offer flexibility.
  \item Always close files, or use the \texttt{with} statement.
  \item Practice reading, processing, and displaying file data.
\end{itemize}
