\chapter{12.6 The \texttt{with} Statement}

\section*{Overview}
A \texttt{with} statement is Python’s built-in way of managing resources — such as files — safely and elegantly.  
When used with \texttt{open()}, it guarantees that the file is closed automatically when the block of code completes,  
even if an exception occurs inside the block.

Think of it as a friendly librarian: you borrow a book (open a file), do your reading,  
and no matter what happens — whether you finish or drop your coffee — the librarian takes the book back and shelves it neatly.

\begin{quote}
\textbf{Docs to Bookmark:}
\begin{itemize}
  \item \href{https://docs.python.org/3/reference/compound_stmts.html#the-with-statement}{The with statement (Python Docs)}
  \item \href{https://docs.python.org/3/tutorial/inputoutput.html#methods-of-file-objects}{File objects in Python}
\end{itemize}
\end{quote}

---

\section{Basic Example: Safe File Reading}

\begin{lstlisting}[language=Python, caption={Opening and reading a file safely.}]
print("Opening myfile.txt")
with open("myfile.txt", "r", encoding="utf-8") as f:
    contents = f.read()
    print(contents)
print("File closed automatically after this block!")
\end{lstlisting}

\noindent
The \texttt{with} statement ensures that once you exit the block, \texttt{f.close()} is called automatically.  
Even if an error occurs inside the block, Python still closes the file properly.

---

\section{Why Not Just Call \texttt{close()} Yourself?}

\begin{lstlisting}[language=Python, caption={Manual open and close -- easy to forget.}]
f = open("oops.txt", "w", encoding="utf-8")
f.write("This file might stay open forever...")
# Oops! We forgot f.close()
\end{lstlisting}

\noindent
If you forget to close a file, it can remain locked or unflushed in memory —  
causing confusion, corrupted files, or grumpy system administrators.  

\begin{lstlisting}[language=Python, caption={Fixed using a with statement.}]
with open("oops.txt", "w", encoding="utf-8") as f:
    f.write("This version closes itself. Much better!")
\end{lstlisting}

---

\section{Reading and Writing Together}

\begin{lstlisting}[language=Python, caption={Using the same file object for reading and writing.}]
print("Opening numbers.txt for reading and writing...")
with open("numbers.txt", "r+", encoding="utf-8") as f:
    num1 = int(f.readline())
    num2 = int(f.readline())
    product = num1 * num2
    f.write(f"\nProduct: {product}\n")
print("Closed numbers.txt automatically.")
\end{lstlisting}

---

\section{Cautionary Tale: The File That Refused to Close}

\begin{lstlisting}[language=Python, caption={Why using `with` is safer.}]
# Scenario 1: Forgetting to close a file manually
try:
    f = open("chaos.txt", "w", encoding="utf-8")
    f.write("Everything is fine so far...")
    raise RuntimeError("Oops! Chaos strikes!")  # Something goes wrong!
    f.close()
except Exception as e:
    print("Caught error:", e)
    print("Did we close the file? Nope - it's still open!")

# Scenario 2: Using 'with' guarantees closure even after errors
try:
    with open("calm.txt", "w", encoding="utf-8") as f:
        f.write("Tranquility restored. File will close no matter what.")
        raise RuntimeError("A wild exception appears!")
except Exception as e:
    print("Caught safely:", e)
    print("File 'calm.txt' has already been closed automatically.")
\end{lstlisting}


\noindent
Even though both blocks raise errors, only the second one guarantees that the file is properly closed.

---

\section{Challenge Example: Writing Logs}

\begin{lstlisting}[language=Python, caption={Appending a timestamped log entry safely.}]
from datetime import datetime

def log_event(message):
    """Append an event to a log file with timestamp."""
    timestamp = datetime.now().strftime("%Y-%m-%d %H:%M:%S")
    with open("log.txt", "a", encoding="utf-8") as log:
        log.write(f"[{timestamp}] {message}\n")

log_event("Program started successfully.")
log_event("User pressed the red button.")
\end{lstlisting}

---

\section{Quiz Questions}
\begin{enumerate}
  \item When does Python automatically close a file opened with \texttt{with}?  
        \textbf{Answer:} When the indented block ends, even if an error occurs.
  \item What is a \texttt{context manager}?  
        \textbf{Answer:} An object that sets up and tears down resources automatically.
  \item What happens if you forget \texttt{close()}?  
        \textbf{Answer:} The file may remain locked or data may not be saved.
\end{enumerate}

---

\section*{Summary}
\begin{itemize}
  \item Use \texttt{with open(...)} to ensure automatic closure of files.
  \item It’s cleaner, safer, and preferred in all modern Python code.
  \item Works great for other resources too (network sockets, database connections, etc.).
\end{itemize}

\begin{quote}
\textbf{Moral of the story:}  
If your program opens files without \texttt{with}, it’s like leaving the refrigerator door open — it still “works,” but you’re wasting power and asking for trouble.
\end{quote}
