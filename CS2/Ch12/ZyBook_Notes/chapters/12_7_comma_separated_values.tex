\chapter{12.7 Comma-Separated Values (CSV) Files}

\section*{Overview}
CSV (Comma-Separated Values) files are one of the most common ways to store and exchange data between programs.  
They’re simple text files where each line represents a row of data, and each value is separated by a comma.  
Think of them as a poor man’s spreadsheet — small, scrappy, and easy to work with!

\begin{quote}
\textbf{Learning goals:}
\begin{itemize}
  \item Understand how data is structured inside a CSV file.
  \item Learn to read and process CSV data using Python’s built-in \texttt{csv} module.
  \item Practice writing data back into CSV format.
  \item Perform calculations on tabular data read from files.
\end{itemize}
\end{quote}

---

\section{A Typical CSV File}

\begin{lstlisting}[language=Python, caption={Sample contents of a CSV file.}]
name,hw1,hw2,midterm,final
Petr Little,9,8,85,78
Sam Tarley,10,10,99,100
Jeff King,4,2,55,61
\end{lstlisting}

Each line represents a student record. Each comma separates a value (called a \textbf{field}).  
Python can read and process these values efficiently using the \texttt{csv} module.

---

\section{Reading Rows from a CSV File}

\begin{lstlisting}[language=Python, caption={Reading each row in a CSV file.}]
import csv

with open("grades.csv", "r", encoding="utf-8") as csvfile:
    grades_reader = csv.reader(csvfile, delimiter=",")
    row_num = 1
    for row in grades_reader:
        print(f"Row #{row_num}: {row}")
        row_num += 1
\end{lstlisting}

\textbf{Explanation:}
\begin{itemize}
  \item \texttt{csv.reader()} returns an iterator — one list of fields per line.
  \item The optional \texttt{delimiter} argument can change how fields are split (e.g., use \texttt{;} or \texttt{|}).
  \item Each row is a list of strings. You can convert them to numbers later.
\end{itemize}

---

\section{Calculating with CSV Data}

\begin{lstlisting}[language=Python, caption={Using CSV data to compute averages.}]
import csv

grades = {}

with open("grades.csv", "r", encoding="utf-8") as csvfile:
    reader = csv.reader(csvfile)
    header = next(reader)  # Skip the first row
    for row in reader:
        name = row[0]
        scores = [float(x) for x in row[1:]]
        hw1, hw2, midterm, final = scores
        final_score = (hw1/10 * 0.05 + hw2/10 * 0.05 +
                       midterm/100 * 0.40 + final/100 * 0.50) * 100
        grades[name] = final_score

for name, score in grades.items():
    print(f"{name} earned {score:.1f}%")
\end{lstlisting}

\textbf{Output Example:}
\begin{verbatim}
Petr Little earned 81.5%
Sam Tarley earned 99.6%
Jeff King earned 55.5%
\end{verbatim}

---

\section{Writing to a CSV File}

\begin{lstlisting}[language=Python, caption={Writing lists of data to a CSV file.}]
import csv

row1 = ["100", "50", "29"]
row2 = ["76", "32", "330"]

with open("grades_output.csv", "w", newline="", encoding="utf-8") as csvfile:
    writer = csv.writer(csvfile)
    writer.writerow(row1)
    writer.writerow(row2)

print("Data successfully written to grades_output.csv")
\end{lstlisting}

\textbf{Notes:}
\begin{itemize}
  \item Always include \texttt{newline=""} when writing CSV files on Windows to prevent blank lines.
  \item \texttt{writerow()} writes one list at a time.
  \item \texttt{writerows()} writes multiple rows from a list of lists.
\end{itemize}

---

\section{Mini Challenge}

Create a script called \texttt{combine\_grades.py} that reads two CSV files:  
\texttt{midterm.csv} and \texttt{final.csv}, merges the data by student name,  
and writes a combined report to \texttt{summary.csv}.

\begin{lstlisting}[language=Python, caption={Combining two CSV files.}]
import csv

with open("midterm.csv", "r") as mid, open("final.csv", "r") as fin, open("summary.csv", "w", newline="") as out:
    mid_reader = csv.reader(mid)
    fin_reader = csv.reader(fin)
    writer = csv.writer(out)

    next(mid_reader), next(fin_reader)  # skip headers
    writer.writerow(["Name", "Midterm", "Final", "Average"])

    for mid_row, fin_row in zip(mid_reader, fin_reader):
        name = mid_row[0]
        mid_score = float(mid_row[1])
        fin_score = float(fin_row[1])
        avg = (mid_score + fin_score) / 2
        writer.writerow([name, mid_score, fin_score, f"{avg:.1f}"])
\end{lstlisting}

---

\section*{Summary}
\begin{itemize}
  \item CSV files store tabular data using commas as separators.
  \item Use the \texttt{csv} module for reading and writing structured data safely.
  \item Always use \texttt{with open(...)} to handle file I/O automatically.
  \item Convert strings to \texttt{int} or \texttt{float} before performing calculations.
  \item Include \texttt{newline=""} when writing CSV files to avoid spacing issues.
\end{itemize}
