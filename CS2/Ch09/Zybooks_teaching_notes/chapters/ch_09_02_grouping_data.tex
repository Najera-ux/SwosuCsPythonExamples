\chapter{Classes: Grouping Data}

\section{Why Group Data into Classes?}

Many variables in a program are closely related and should be bundled together.
For instance, a time value consists of hours and minutes.
Instead of managing two separate variables, we can define a \textbf{class} that groups them into one logical unit.

\begin{quote}
A \textbf{class} defines a new data type that groups related data (called \emph{attributes}) and the operations that act on them (called \emph{methods}).
\end{quote}

\section{Constructing a Simple Class}

\subsection*{The \texttt{class} Keyword}

\begin{verbatim}
class ClassName:
    # Statement-1
    # Statement-2
    # ...
    # Statement-N
\end{verbatim}

A class defines both the data and the behaviors of an object.
The example below defines a class named \texttt{Time} with two attributes.

\subsection*{Example: Defining a Class with Two Data Attributes}

\begin{verbatim}
class Time:
    """A class that represents a time of day."""
    def __init__(self):
        self.hours = 0
        self.minutes = 0
\end{verbatim}

Here, the \texttt{\_\_init\_\_()} function is a special method called a \textbf{constructor}.
It runs automatically when a new object (or instance) of \texttt{Time} is created.

\section{Creating and Using an Object}

\begin{verbatim}
my_time = Time()
my_time.hours = 7
my_time.minutes = 15

print(f"{my_time.hours} hours and {my_time.minutes} minutes")
\end{verbatim}

\textbf{Output:}
\begin{verbatim}
7 hours and 15 minutes
\end{verbatim}

Each variable created from the class (\texttt{my\_time}) is called an \textbf{instance}.
The attributes of that instance are accessed using the \textbf{dot operator} (\texttt{.}).

\section{Multiple Instances of a Class}

You can create multiple independent instances, each maintaining its own data.

\begin{verbatim}
time1 = Time()
time1.hours = 7
time1.minutes = 30

time2 = Time()
time2.hours = 12
time2.minutes = 45

print(f"{time1.hours} hours and {time1.minutes} minutes")
print(f"{time2.hours} hours and {time2.minutes} minutes")
\end{verbatim}

\textbf{Output:}
\begin{verbatim}
7 hours and 30 minutes
12 hours and 45 minutes
\end{verbatim}

\section{Key Terms}

\begin{description}
    \item[class] A grouping of related data and behaviors.
    \item[attribute] A variable stored within a class or instance.
    \item[method] A function that belongs to a class.
    \item[\_\_init\_\_] The constructor method called automatically when creating a new object.
    \item[self] Refers to the instance itself within class methods.
    \item[instance] An individual object created from a class.
\end{description}

\section{Practice Activity}

\subsection*{Activity 9.2.1 – Define and Instantiate a Class}

\begin{verbatim}
class Person:
    def __init__(self):
        self.name = ""

person1 = Person()
person1.name = "Van"
print(f"This is {person1.name}")
\end{verbatim}

\textbf{Output:}
\begin{verbatim}
This is Van
\end{verbatim}

\subsection*{Activity 9.2.2 – Create Your Own Class}

Define a class called \texttt{BookData} with three attributes:
\texttt{year\_published}, \texttt{title}, and \texttt{num\_chapters}.
Create an instance of the class and assign values to its attributes.

\begin{verbatim}
class BookData:
    def __init__(self):
        self.year_published = 0
        self.title = "Unknown"
        self.num_chapters = 0

my_book = BookData()
my_book.year_published = 2001
my_book.title = "A Tale of Two Cities"
my_book.num_chapters = 45

print(f"{my_book.title} ({my_book.year_published}) has {my_book.num_chapters} chapters.")
\end{verbatim}

\textbf{Output:}
\begin{verbatim}
A Tale of Two Cities (2001) has 45 chapters.
\end{verbatim}

\section*{Reflection Questions}
\begin{enumerate}
    \item What is the difference between a class and an instance?
    \item Why is the \texttt{self} keyword required in class definitions?
    \item How does \texttt{\_\_init\_\_()} help organize data?
\end{enumerate}

