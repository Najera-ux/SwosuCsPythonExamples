\chapter{Class Constructors}
\label{ch:class_constructors}

\section*{9.6 Overview}
A class constructor is a special method that defines how new objects are created and initialized.
In Python, the constructor method is named \texttt{\_\_init\_\_()}.
Constructors are used to set up instance attributes and can accept parameters to configure each new object.

---

\section*{Adding Parameters to a Constructor}

\begin{lstlisting}[language=Python, caption={A simple constructor with parameters.}]
class RaceTime:
    def __init__(self, start_time, end_time, distance):
        """Initialize race data."""
        self.start_time = start_time    # Format: "H:MM"
        self.end_time = end_time
        self.distance = distance        # In miles

# Create RaceTime objects
time_jason = RaceTime("3:15", "7:45", 26.21875)
time_bobby = RaceTime("3:15", "6:30", 26.21875)
\end{lstlisting}

Here, each new object receives its starting time, ending time, and race distance.
This is more powerful than setting everything to zero — each instance can have its own data immediately upon creation.

---

\section*{Complete RaceTime Example}

\begin{lstlisting}[language=Python, caption={RaceTime class with methods.}]
class RaceTime:
    def __init__(self, start_hrs, start_mins, end_hrs, end_mins, dist):
        self.start_hrs = start_hrs
        self.start_mins = start_mins
        self.end_hrs = end_hrs
        self.end_mins = end_mins
        self.distance = dist

    def print_time(self):
        if self.end_mins >= self.start_mins:
            minutes = self.end_mins - self.start_mins
            hours = self.end_hrs - self.start_hrs
        else:
            minutes = 60 - self.start_mins + self.end_mins
            hours = self.end_hrs - self.start_hrs - 1
        print(f"Time to complete race: {hours}:{minutes:02d}")

    def print_pace(self):
        total_minutes = (self.end_hrs * 60 + self.end_mins) - \
                        (self.start_hrs * 60 + self.start_mins)
        pace = total_minutes / self.distance
        print(f"Average pace: {pace:.2f} mins/mile")

# Example interaction
distance = 5.0
start_hrs = int(input("Enter starting time hours: "))
start_mins = int(input("Enter starting time minutes: "))
end_hrs = int(input("Enter ending time hours: "))
end_mins = int(input("Enter ending time minutes: "))

race_time = RaceTime(start_hrs, start_mins, end_hrs, end_mins, distance)
race_time.print_time()
race_time.print_pace()
\end{lstlisting}

---

\section*{Default Constructor Parameters}

Constructors can also include default values for convenience.  
This reduces repetition and helps when typical defaults are common.

\begin{lstlisting}[language=Python, caption={Employee class with default parameters.}]
class Employee:
    def __init__(self, name, wage=8.25, hours=20):
        """Default employee works part-time and earns minimum wage."""
        self.name = name
        self.wage = wage
        self.hours = hours

employees = [
    Employee("Todd"),                   # uses defaults
    Employee("Jason"),                  # uses defaults
    Employee("Tricia", 12.50, 40)       # manager: custom values
]

for e in employees:
    print(f"{e.name} earns ${e.wage * e.hours:.2f} per week")
\end{lstlisting}

---

\section*{Constructors in Practice: Student Example}

\begin{lstlisting}[language=Python, caption={Constructor with several defaults.}]
class Student:
    def __init__(self, name, grade=9, honors=False, athletics=False):
        self.name = name
        self.grade = grade
        self.honors = honors
        self.athletics = athletics

johnny = Student("Johnny", grade=11, honors=True)
tommy = Student("Tommy")

print(f"{johnny.name}: grade {johnny.grade}, honors={johnny.honors}")
print(f"{tommy.name}: grade {tommy.grade}, athletics={tommy.athletics}")
\end{lstlisting}

---

\section*{Constructor Exercises}

\begin{enumerate}
    \item Modify \texttt{Employee} so that it tracks yearly pay in addition to hourly.
    \item Add a method \texttt{is\_manager()} that returns \texttt{True} if wage > 12.
    \item Rewrite \texttt{RaceTime} so that it accepts total minutes instead of hours and minutes separately.
\end{enumerate}

